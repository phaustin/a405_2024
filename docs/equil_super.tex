\documentclass[12pt]{article}
\usepackage{xr,geometry,fancyhdr,hyperref,ifpdf,rcs}
\usepackage{color,lastpage,Ventry,url,paunits,shortcuts,smallsec,float,amsmath}
\geometry{letterpaper,top=50pt,hmargin={20mm,20mm},headheight=15pt} 

\pagestyle{fancy} 

\let\rlh\rightleftharpoons
\RCS $Revision: 1.1 $
\RCS $Date: 2002/01/11 03:49:21 $

\fancypagestyle{first}{
\lhead{\textbf{C535}}
\chead{\textbf{Notes on time constants}} 
\rhead{p.~\thepage/\pageref{LastPage}}
\lfoot{} 
\cfoot{} 
\rfoot{}
}

\ifpdf
    \usepackage[pdftex]{graphicx} 
    \usepackage{hyperref}
    \pdfcompresslevel=0
    \DeclareGraphicsExtensions{.pdf,.jpg,.mps,.png}
\else
    \usepackage{hyperref}
    \usepackage[dvips]{graphicx}
    \DeclareGraphicsRule{.eps.gz}{eps}{.eps.bb}{`gzip -d #1}
    \DeclareGraphicsExtensions{.eps,.eps.gz}
\fi
\externaldocument[lec1,]{lec1}
\externaldocument[lec2,]{lec2}
\externaldocument[lec3,]{lec3}
\hyperbaseurl{http://www.eos.ubc.ca/courses/atsc405/}

\begin{document}
\pagestyle{first}

\begin{center}
ATSC 405 Summary:  Time constants for droplet growth\\
\end{center}



\section{Timescale for the droplet growth equation (Lohmann page 204)}
\label{sec:drop}

Suppose we want to determine how the time-scale for
the response of an individual droplet to a change in supersaturation
compares with the time-scale for the change in supersaturation for the
ascending parcel.  In other words, are the droplets adjusting rapidly
to changing conditions, or are they lagging?

To construct a time-scale 
start with the simplified droplet growth equation (Lohmann eq. 7.8):

\begin{equation}
  \label{finala}
  \frac{dm}{dt} = 4 \pi \rho_l r^2 \frac{dr}{dt} = 4 \pi r D (\rho_{v \infty} - \rho_{v r})
\end{equation}

If we neglect the solution and curvature terms then $\rho_{v r}=\rho_{v s}$,  dividing through
by $\rho_{v s}$ gives:

\begin{equation}
  \label{finalb}
  \frac{1}{r}\frac{dr}{dt} = \frac{D\, \rho_{v s}}{r^2 \rho_l} \left (\frac{\rho_{v \infty}}{\rho_{v s}} - 1 \right )
      = \frac{D\,\rho_{v s}}{r^2 \rho_l } \left (\frac{e}{e_s} - 1 \right ) =\frac{D\,\rho_{v s}}{r^2 \rho_l }(S - 1)
\end{equation}
What have I gained from doing this?  Notice that the rhs of
(\ref{finalb}) has units of 1/time.  If I freeze the rhs by selecting
a particular radius $r_0$ and saturation S, then I can consider the
rhs constant and  write it in terms of a time scale as $1/\tau$, where

\begin{equation}
  \label{eq:tau}
  \frac{dr}{r} = \frac{1}{\tau} \,dt  = \frac{D\,\rho_{v s}\,SS}{r_0^2 \rho_l }\,dt
\end{equation}
where I've written the supersaturation as $SS=(S-1)$.

I can then
integrate (\ref{eq:tau}) to get  a time constant:


\begin{equation}
  \label{eq:integ}
  r = r_0 \exp(t/\tau)
\end{equation}
where $\tau=(r_0^2 \rho_l)/(\rho_{vs}\, D\,SS)$. Note that I need to
remember that (\ref{eq:integ}) is only valid for small perturbations
of $r$ in the neighborhood of $r_0$.

How big is $\tau$?  Choose $\rho_l=1000\ \kgmmm$, $D=2.36 \times 10^{-5}\ \un{m^2\,s^{-1}}$,
$r_0=5 \mum$, $p=9 \times 10^5$ Pa, $T=280\ \un{K}$,
$\rho_{vs}=e_s/(R_v T) = e_s(T)/(R_v \cdot 280\ \un{K}) = 990/(461 \cdot 280\ \un{K})%
=7.7\ \gkg$,  $SS=0.002$, I get: $\tau = 1/1.89 \approx 0.5\ \un{s}$.  This tells
me that 5 \mum droplets respond within about a second to fluctuations in supersaturation or
temperature.  Note that 50 \mum drops are much less responsive, because of the $r_0^2$ term
in the expression for $\tau$.  Substituting $r_0=50\ \mum$ into (\ref{eq:tau}) I get
$\tau \approx 53\ s$.


\section{Timescale for the supersaturation equation}
\label{sec:supersat}

The curve for the supersaturation shown in the Lohmann Figure 7.7:
shows 1) that S decays exponentially after the aerosols activate and 2) that S
approaches an equilibrium value different from 0 in a constant velocity updraft.
It is possible to derive (fairly) simple expressions for both the timescale for
the exponential decay, $\tau$ and the long-time equilibrium value of the supersaturation,
$S_\infty$.

Start with an approximate expression for the mixing ratio:

\begin{equation}
  \label{eq:mix}
  w_v = \frac{\rho_v}{\rho_d} = \frac{R_d}{R_v} \frac{e}{p - e} \approx \frac{R_d}{R_v} \frac{e}{p}=
\epsilon \frac{e}{p} = \frac{S \epsilon e_s}{p} = (1 + SS) \frac{\epsilon e_s}{p}
\end{equation}
where $SS=(S-1)$ defines the supersaturation.

To get an equation for $\frac{d\,SS}{dt}$, we need to differentiate (\ref{eq:mix}).  Do this
and show that using the chain rule gives Lohmann 7.24 and 7.25:

\begin{equation}
  \label{eq:chain}
  \frac{d w_v}{dt} = \left (1 + SS \right )  \left [ \frac{-\epsilon e_s}{p^2} 
\left ( \frac{-g p V}{R_d T} \right ) + \frac{\epsilon}{p} \left ( 
\frac{\epsilon e_s L}{R_d T^2} \right ) \frac{dT}{dt} \right ]
+ \frac{\epsilon e_s}{p} \frac{dSS}{dt}
\end{equation}
where $V$ is the vertical velocity $dz/dt$ and I've used the Clausius-Clapeyron equation and
assumed hydrostatic balance:

\begin{subequations}
  \begin{eqnarray}
    \label{eq:various}
    \frac{de_s}{dT} &=& \frac{\epsilon L e_s}{R_d T^2}\\
    \frac{dp}{dt} & = & - \rho g \frac{dz}{dt} = -\frac{g p}{R_d T} V
  \end{eqnarray}
\end{subequations}

From conservation of moist static energy also have an expression for $\frac{dT}{dt}$ in the ascending parcel:

\begin{equation}
  \label{eq:adia}
  \frac{dT}{dt} = \frac{-L}{c_p} \frac{dw_v}{dt} - \frac{g V}{c_p}
\end{equation}
and as usual from conservation of total water $w_t$ we can relate the change in $w_v$ to the change
in $w_l$:

\begin{equation}
  \label{eq:conserv}
  \frac{d w_v}{dt} = - \frac{d w_l}{dt} = - \frac{d}{dt} \left [ \frac{4}{3} \pi \rho_l
  \int_0^\infty n(r) r^3 dr \right ] = -4 \pi \rho_l \int_0^\infty n(r) r^2 \frac{dr}{dt} dr
\end{equation}
where I neglect the fact that $n(r)$ is really a function of time as well (since the entire
drop size distribution is shifting to larger $r$ during the ascent.

I can rewrite (\ref{finalb}) to get and expression for $r \frac{dr}{dt}$ using the supersaturation:

\begin{equation}
  \label{eq:rdr}
  r \frac{dr}{dt} = \frac{D\,\rho_{v s}}{ \rho_l }(SS) = G\,SS
\end{equation}
where $G=\frac{D  \rho_{v s}}{ \rho_l}$ is assumed constant during the ascent
(not a great assumption, because of the temperature dependence of $\rho_{v s}$)

So substituting (\ref{eq:rdr}) into (\ref{eq:conserv}) gives a new form for the rate of vapor change:

\begin{equation}
  \label{eq:newvap}
    \frac{d w_v}{dt} = - 4 \pi \rho_l G\, SS\, \int_0^\infty n(r) \, r dr
\end{equation}

This simplifies things because the mean droplet radius is defined by:

\begin{equation}
  \label{eq:mean}
\overline{r} = \frac{\int_0^\infty n(r) \, r dr}{\int_0^\infty n(r)  dr} = 
\frac{\int_0^\infty n(r) \, r dr}{N}
\end{equation}
where $N$ is the total droplet number per unit volume.

So finally, we've got a simple expression for the change in $w_v$ wrt the
dropsize distribution:

\begin{equation}
  \label{eq:simple}
  \frac{d w_v}{dt} = - 4\pi \rho_l G\,SS\,N\,\overline{r}
\end{equation}

Plugging  (\ref{eq:adia}) into (\ref{eq:chain}) and rearrange to get:

\begin{equation}
  \label{eq:dSS}
  \frac{d SS}{dt} = - \frac{g V}{R_d T} - \frac{L}{R_v T^2}  \left [- \frac{L}{c_p} \frac{dw_v}{dt}
- \frac{g V}{c_p} \right ] + \frac{p }{\epsilon e_s} \frac{dw_v}{dt}
\end{equation}
where I've made the approximation that $(1 + SS) \approx 1$ in the first term of (\ref{eq:adia}).
The justification for this is the fact that we know that $SS \approx 0.002$ from
our cloud model runs.

Inserting (\ref{eq:simple}) gives:

\begin{equation}
  \label{eq:dSSb}
  \frac{d SS}{dt} =  \frac{g }{T}  \left [ \frac{L}{c_p R_v T} 
- \frac{1}{R_d} \right ] V  - \left [ \frac{p}{e_s \epsilon} + \frac{L^2}{R_v c_p T^2} \right ]
4 \pi \rho_l G N \overline{r} SS
\end{equation}
The physical interpretation of  (\ref{eq:dSSb}) is that the rate of change of SS depends
on two terms: a source ($SS$ increases due to adiabatic cooling) and a sink ($SS$ decreases
due to droplet growth).

Equation (\ref{eq:dSSb}) has the general form of an exponential decay with timescale $\tau$
to a limiting value $SS_\infty$:

\begin{equation}
  \label{eq:general}
  \frac{d SS}{dt} = - \frac{SS}{\tau} + C
\end{equation}
where:

\begin{subequations}
\label{eq:coeffs}
  \begin{eqnarray}
    \label{eq:various2}
    1/\tau &=& \left [ \frac{p}{e_s \epsilon} + \frac{L^2}{R_v c_p T^2} \right ]
                    4 \pi \rho_l G N \overline{r} \\
    C &=& \frac{g }{T}  \left [ \frac{L}{c_p R_v T} 
            - \frac{1}{R_d} \right ] V
  \end{eqnarray}
\end{subequations}

To solve (\ref{eq:general}) multiply it by the integrating factor $\exp(t/\tau)$:

\begin{equation}
  \label{eq:general2}
\exp(t/\tau)  \frac{d SS}{dt} + \exp(t/\tau) \frac{SS}{\tau} =   C \exp(t/\tau)
\end{equation}
and use the chain rule:

\begin{equation}
  \label{eq:generalb}
 \frac{d}{dt} \left ( \exp(t/\tau) \, SS \right ) =   C \exp(t/\tau)
\end{equation}
Starting at $t=0$ and $SS=SS_0$ integrate (\ref{eq:generalb}) to get

\begin{equation}
  \label{eq:generalc}
SS(t) = (SS_0 - C \tau) \exp \left (- \frac{t}{\tau} \right )  + C \tau
\end{equation}
Since $C\,\tau$ is the value of $SS(t)$ at $t=\infty$, it's clearer to
give it the symbol $S_\infty$ for the equilibrium supersaturation in the updraft:

\begin{equation}
  \label{eq:generald}
SS(t) = (SS_0 - S_{\infty}) \exp \left (- \frac{t}{\tau} \right )  + S_{\infty}
\end{equation}


What does (\ref{eq:generalc}) tell us about the supersaturation history? Looking at
(\ref{eq:coeffs}), we can see:  1) that the timeconstant $\tau$ decreases as
$N \overline{r}$ increases.  We can think of $N \overline{r}$ (also called the ``integral radius'')
as a measure of the ability of droplets to decrease the vapor mixing ratio.  2)
the assymptotic value $SS_\infty$ increases with the updraft velocity $V$ (which sets
the cooling rate).

Choosing $V=0.2\ \ms$, $N=80\ \ccm$ and $\overline{r}=5\ \mum$, with $T$, $p$ as before
I get a time constant for the supersaturation decay of $\tau \approx 3.5\ \un{s}$, and
an assymptotic supersaturation of $SS_\infty=0.04\%$ (my neglect of thermal diffusion
makes these wrong by about a factor of 2, see Rogers and Yau Chapter 7 for the full
treatment).  The fact that the supersaturation and moderately-sized cloud droplets have
about the same response times means that you do have to solve the temperature, vapor,
and droplet growth equations simultaneously.


What would the next step be?  In order to decide what to include next
in the droplet growth model, we would need to calculate time/space
scales for other processes we're currently neglecting:
1) fluctuations in the vertical velocity  2) mixing between parcel
and environment  3) radiative cooling at the edges of the cloud
4) droplet sedimentation and turbulent trajectories  5) the
effect of organic solutes on the equilibrium supersaturation
6) the effect of black carbon on the absorption of solar
radiation by the droplet 7) etc.

\end{document}

%matlab script
%T=280;
%p=90000.
%g=9.8;
%esatPa(T)
%Rv=461.;
%Rd=287.;
%rhovs=esatPa(T)/(Rv*T)
%D=2.36e-5;
%r=5.e-6;
%rhol=1000.;
%rho=p/(Rd*T);
%SS=0.002;
%onetau=D*SS*rhovs/(r^2*rhol*rhovs)
%tau=1./onetau
%eps=0.622
%V=0.2 %updraft in m/s

%es=esatPa(T)
%L=2.5e6
%cp=1004
%G=D*rhovs/rhol
%N=80.e6
%rbar=5.e-6
%newonetau=(p/(es*eps) + L^2./(Rv*cp*T^2))*4*pi*rhol*G*N*rbar
%newtau=1/newonetau

%%Yau Fig 7.2
%Q1a = eps*L*g/(Rd*cp*T);
%Q1b= g/Rd;
%Q1=(1./T)*(Q1a - Q1b)
%Q2a=(Rd*T/(eps*es));
%Q2b=eps*L^2./(p*T*cp);
%Q2=rho*(Q2a + Q2b)
%%P=80000
%%T=253
%%Q2 =  274.3914
%%Q1 =  6.9126e-04
%%Q2 = 1.2507e+03

%%PA 
%Q1a = L/(Rv*cp*T);
%Q1b= 1/Rd;
%Q1=(g/T)*(Q1a - Q1b)
%Q2=(p/(es*eps)) + (L^2/(Rv*cp*T^2))
%SSonetau=Q2*4*pi*rhol*G*N*rbar
%SStau=1/SSonetau
%C=Q1*V
%SSinf=C*SStau


%drdt=D*rhovs/(rhol*r)*0.001

%dwvdt = -4*pi*rhol*G*SS*1.e8*8.e-6

%%% Local Variables:
%%% mode: latex
%%% TeX-master: t
%%% End:
