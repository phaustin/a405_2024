\documentclass[12pt]{article}
\usepackage{color,xr,lastpage,geometry,fancyhdr,ifpdf,amsmath,rcs,paunits,shortcuts}
\pagestyle{fancy}

\lhead{A405}
\chead{Day 28 problems} 
\rhead{\thepage/\pageref{LastPage}}
\lfoot{}
\rfoot{}
\cfoot{}

\geometry{letterpaper,top=50pt,hmargin={20mm,20mm},headheight=15pt} 

\definecolor{darkorange}{rgb}{.71,0.21,0.01}
\definecolor{darkgreen}{rgb}{.12,.54,.11}

% This helps prevent overly long lines that stretch beyond the margins

\ifpdf
  \usepackage[pdftex,  % needed for pdflatex
   breaklinks=true,  % so long urls are correctly broken across lines
   colorlinks=true,
   urlcolor=blue,
   linkcolor=darkorange,
   citecolor=darkgreen,
    ]{hyperref}
    \usepackage[pdftex]{graphicx} 
    \pdfcompresslevel=0
    \DeclareGraphicsExtensions{.pdf,.jpg,.mps,.png}
\else
    \usepackage{hyperref}
    \usepackage[dvips]{graphicx}
    \DeclareGraphicsRule{.eps.gz}{eps}{.eps.bb}{`gzip -d #1}
    \DeclareGraphicsExtensions{.eps,.eps.gz}
\fi

\sloppy



\begin{document}

\begin{center}\textbf{
W\&H problems 6.21, 6.23 and 6.24}
\end{center}

\begin{enumerate}


\item Wallace and Hobbs 6.21 (height)

  \begin{gather}
    \frac{ dh}{dt}  = w - v \label{one} \\
  v = \frac{ 2 g \rho_l r^2}{9 \eta} \label{two}\\
r \frac{ dr}{dt}  = G_l S \\
\Rightarrow \frac{ 1}{2}\frac{dr^2 }{dt}   = G_l S \\
\intertext{integrate from $(t^\prime=0,r^\prime=0)$ to $(t^\prime=t,r^\prime = r$)}
r^2 = 2 G_l S t \label{two} \\
\intertext{insert (\ref{two}) into (\ref{one}) and integrate:}
    \frac{ dh}{dt}  = w - \frac{ 2 g \rho_l 2 G_l S t}{9 \eta} \label{three} \\
\int_0^h  dh^\prime = \int_0^t \left [ w - \frac{ 2 g \rho_l 2 G_l S t^\prime}{9 \eta} \right ] dt^\prime \\
h= wt - \frac{ 2 g \rho_l  G_l S t^2}{9 \eta}
  \end{gather}

\item Wallace and Hobbs 6.23 (drizzle)

We know from the sweep-out equation (6.26) that

\begin{gather}
  \frac{ dm}{dt}  = \pi r^2 V E lwc
\end{gather}
where $lwc$ is the liquid water content in \un{kg\,m^{-3}}.

Inserting $m=\frac{4 }{3}  \pi r^3 \rho_l$:

 \begin{gather}
   \rho_l \frac{ 4}{3} \pi r^2 \frac{dr}{dt}   = \pi r^2 V E lwc \\
 \frac{ dr}{dt} = \frac{ V E}{4 \rho_l} lwc
 \end{gather}
For the lwc, we have 100 droplets \cc with $\overline{r }= 10^{-3}$ cm,which gives a lwc of $4/3 \pi 1. \times 10^{-7}$ \un{g\,cm^{-3}} .  Taking
E=0.8 and $V=6000 r$ cm/s, with r in cm:

\begin{gather}
  \frac{ dr}{dt} = \frac{ 6000 \times r 4/3 \pi  10^{-7} \times 0.8}{4} \\
\frac{ dr}{r} = 1.6 \pi 10^{-4} dt \\
\text{integrating both sides} \nonumber 
\int_{0.01\ cm}^{0.1\ cm} = 1.6 \pi 10^{-4} \int_0^t dt = 1.6 \pi 10^{-4} t \\
\frac{\ln 0.1 - \ln 0.01 }{1.6 \pi 10^{-4}} = t \\
t = 4579\ seconds = 76.3\ minutes
\end{gather}


\item WhH 6.24 -- see web page for matlab solution

\end{enumerate}

\end{document}

%%% Local Variables:
%%% mode: latex
%%% TeX-master: t
%%% End:
