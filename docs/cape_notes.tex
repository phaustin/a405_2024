\documentclass[12pt]{article}
\usepackage{geometry,fancyhdr,xr,hyperref,ifpdf,amsmath,rcs,indentfirst}
\usepackage{lastpage,longtable,Ventry,url,paunits,shortcuts,smallsec,color,tightlist,float}
\geometry{letterpaper,top=50pt,hmargin={20mm,20mm},headheight=15pt} 
\usepackage[stable]{footmisc}

\pagestyle{fancy} 

\RCS $Revision: 1.5 $
\RCS $Date: 2002/01/09 03:50:54 $

\fancypagestyle{first}{
\lhead{Maximum vertical velocity and CAPE}
\chead{}
\rhead{page~\thepage/\pageref{LastPage}}
\lfoot{} 
\cfoot{} 
\rfoot{}
}

\ifpdf
    \usepackage[pdftex]{graphicx} 
    \usepackage{hyperref}
    \pdfcompresslevel=0
    \DeclareGraphicsExtensions{.pdf,.jpg,.mps,.png}
\else
    \usepackage{hyperref}
    \usepackage[dvips]{graphicx}
    \DeclareGraphicsRule{.eps.gz}{eps}{.eps.bb}{`gzip -d #1}
    \DeclareGraphicsExtensions{.eps,.eps.gz}
\fi

\newcommand{\dbar}{d\mkern-6mu\mathchar'26} 
\begin{document}
\newcommand{\vect}[1]{\boldsymbol{\vec{#1}}}
\pagestyle{first}


\section{CAPE and w(z)}
\label{sec:cape}

Some notes on convective available potential energy, building Thompkins and \href{https://www.eoas.ubc.ca/books/Practical_Meteorology/}{Stull: Practical Meteorology}
Chapter 14 pp. 503-508

Recall the equation for vertical velocity from the \href{https://www.dropbox.com/scl/fi/ygb2bi2riqo23ostxo8lw/buoyancy.pdf?rlkey=b80rbwtzartk4qp5dt9gjvsf6&dl=0}{buoyancy notes}

\begin{equation}
  \label{eq:newton4}
 \frac{dw}{dt} = -\frac{1}{\overline{\rho}} \frac{\partial p^\prime}{\partial z}
- g \left ( \frac{\rho^\prime}{\overline{\rho}} \right )
\end{equation}
where we'll neglect the pressure acceleration term and define the second term as the
the buoyancy $B$:

\begin{equation}
  \label{eq:buoyancy}
  B = - g \left ( \frac{\rho^\prime}{\overline{\rho}} \right )
\end{equation}

Suppose I can follow a parcel as it ascends from the surface, and measure the vertical velocity $w$ as a function of height and time.   I can then use the chain rule on (\ref{eq:newton4}) and get:

\begin{equation}
  \label{eq:newton5}
 \frac{dw(z)}{dt} = \frac{dw}{dz} \frac{dz}{dt} = B
\end{equation}


But $\frac{dz}{dt} = w$ so rearranging:

\begin{equation}
  \label{eq:newton6}
 \frac{dw(z)}{dt} = w \frac{dw}{dz} = B
\end{equation}
which can be rewritten as:

\begin{equation}
  \label{eq:newton7}
  w\,dw  = B dz
\end{equation}

Integrating (\ref{eq:newton7}) from the surface where $w = 0$ to a height $z^\prime$ gives:

\begin{equation}
  \label{eq:newton7}
  \frac{1}{2} w^2 =  \int_0^{z^\prime} B dz = \text{CAPE}
\end{equation}

So

\begin{equation}
  \label{eq:cape}
  w(z^\prime) =  \sqrt{2\,\text{CAPE}}
\end{equation}

It's also true that $B\,dz$ is force $\times$ distance = work,  and $(1/2) w^2$ is the kinetic energy per kg, so that (\ref{eq:cape}) is assuming that every bit of work done by buoyancy on the parcel is used to accelerate it to velocity $w(z^\prime)$.


\end{document}



%%% Local Variables:
%%% mode: latex
%%% TeX-master: t
%%% End:
