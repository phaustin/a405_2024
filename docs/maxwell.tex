

\documentclass[hyperref={colorlinks=true,linkcolor=blue,urlcolor=blue},numbers]{beamer}
\usepackage{amsmath,paunits,shortcuts}
%\usepackage[usenames]{color}
\setbeamertemplate{navigation symbols}{}
\setbeamertemplate
    {footline}
    {\quad\strut\insertsection
      \hfill\insertframenumber/\inserttotalframenumber\strut\quad} 

\setbeamertemplate{blocks}[rounded][shadow=true]
\setbeamercovered{transparent}


\begin{document}


\begin{frame}
  \frametitle{Maxwell's relations -- why does $g_l = g_v$ for phase change}

Here's some extra perspective on Lohmann Chapter 2, specifically the fact that 2 phases of water in equilibrium must have equal Gibbs free energies.

Start with the basic first law using $u$ and $\alpha$:

  \begin{subequations}
  \begin{gather}
    \text{Physics -- first law, reversible: }\nonumber \\
Tds = du  + p d \alpha \label{eq:firstA}\\
\text{Math -- total differential of two variables: } \nonumber\\
ds = \left ( \frac{\partial s }{\partial u}   \right )_\alpha du + 
\left (  \frac{ \partial s}{ \partial \alpha}  \right )_u d\alpha \label{eq:mathA}\\
\text{line up the terms: for (\ref{eq:firstA}) and (\ref{eq:mathA})} \nonumber\\
\left ( \frac{\partial s }{\partial u}   \right )_\alpha = \frac{ 1}{T} \label{eq:temp}\\
\left (  \frac{ \partial s}{ \partial \alpha}  \right )_u = \frac{p }{T}\label{eq:press} 
  \end{gather}
  \end{subequations}
\end{frame}

\begin{frame}
  \frametitle{Thermal contact means equal temperatures}
Now use them:  consider two containers in thermal contact which
allows them to exchange internal energy $u$ but not volume $\alpha$:

For the total system a + b, $u_{T} = u_a + u_b$ is a constant and

\begin{gather*}
  ds_{T} = \left ( \frac{\partial s_T }{\partial u_A}   \right ) du_a
+ \left ( \frac{\partial s_T }{\partial u_b} \right ) du_b
\end{gather*}

We know that energy is conserved, so $du_a = -du_b$.  We also know that
at equilibrium the total entropy is maximum, so $ds_T = 0$ and:

\begin{gather}
  \left ( \frac{\partial s_T }{\partial u_A} \right ) 
= \left ( \frac{\partial s_T }{\partial u_b} \right )\nonumber\\
\text{or using (\ref{eq:temp}): } \frac{ 1}{T_a}  = \frac{ 1}{T_b} 
\label{eq:thetemp}
\end{gather}
\textit{\textbf{So two systems in thermal contact have the same temperature}}

\end{frame}

\begin{frame}
  \frametitle{Mechanical contact means equal pressures}
Now flip this:  consider two containers in mechanical contact at the
same temperature so they can exchange volume $\alpha$:

For the total system a + b, $\alpha_{T} = \alpha_a + \alpha_b$ is a constant and

\begin{gather*}
  ds_{T} = \left ( \frac{\partial s_T }{\partial \alpha_A}   \right ) d\alpha_a
+ \left ( \frac{\partial s_T }{\partial \alpha_b} \right ) d\alpha_b
\end{gather*}

We know that volume is conserved, so $d\alpha_a = -d\alpha_b$.  
We also know that
at equilibrium the total entropy is maximum, so $ds_T = 0$ and:

\begin{gather}
  \left ( \frac{\partial s_T }{\partial \alpha_A} \right ) 
= \left ( \frac{\partial s_T }{\partial \alpha_b} \right )\nonumber\\
\text{or using (\ref{eq:press}): } \frac{ p_a}{T}  = \frac{ p_b}{T}
\label{eq:press} 
\end{gather}

\textit{\textbf{So two systems in mechanical contact have the same pressure}}

\end{frame}

\begin{frame}
  \frametitle{Diffusive contact means equal (specific) Gibbs free energies/kg}

  \begin{itemize}
  \item But what if your systems are simultaneously exchanging volume and energy?
This isn't a problem, because $s$ is a state variable and path independent.
So take system through two steps: one changing energy at constant
volume, and one change volume at constant energy.

\item What if we are exchanging molecules as well?  Now we need to add the
variation of number of water molecules to
the differential.  First, take a look at the first law if we
are adding  $dN$ molecules, each with energy $\mu$:

\begin{equation}
\label{eq:chem}
  dU = T dS - PdV + \mu dN
\end{equation}
$\mu$ is called the ``chemical potential''.  Note the capital letters, from here on
we need to track extensive variables because we are tracking movement of molecules.
  \end{itemize}

\end{frame}

\begin{frame}
  \frametitle{Diffusive equilibrium II}

If there are several
different types of molecule, each with a different  $\mu_i$, then

\begin{equation}
\label{eq:chem2}
  dU = T dS - PdV + \sum \mu_i dN_i
\end{equation}


Rearranging (\ref{eq:chem2}):

\begin{gather}
\text{physics: } dS = \frac{1 }{T} dU + \frac{ P}{T} dV - \frac{ \mu}{T} dN\\
  \text{Math: } dS = \left ( \frac{\partial S }{\partial U} \right ) dU +
\left ( \frac{\partial S }{\partial V} \right ) dV +
\left ( \frac{\partial S }{\partial N} \right ) dN
\end{gather}
\end{frame}

\begin{frame}
  \frametitle{Diffusive equilibrium III}

  \begin{subequations}
  \begin{gather}
\text{As before, line up the terms and read off: } \nonumber\\
 \left ( \frac{\partial S }{\partial U}   \right )_{V,N} = \frac{ 1}{T} \\
 \left (  \frac{ \partial S}{ \partial V}  \right )_{U,N} = \frac{p }{T} \\
 \left (  \frac{ \partial S}{ \partial N}  \right )_{U,V} = -\frac{\mu }{T}
\label{eq:diff3}
\end{gather}
  \end{subequations}

Using the same approach as with thermal and mechanical contact, we can show
that vapor and liquid separated by an interface at constant U and V will
maximize entropy when $\mu_v = \mu_l$.

\end{frame}

\begin{frame}
  \frametitle{Change to Gibbs free energy for constant T and P}

  \begin{subequations}
  \begin{gather}
\text{physics: }    G = H - TS\  \text{(definition)}\label{eq:one}\\
\text{math: }     dG = dH - T dS - S dT \text{ (differential of G)} \label{eq:two}\\
\text{physics: } T dS = dH - V dP - \mu dN \text{ (first law (reversible))} \label{eq:three}\\
\text{math: } dG = V dP - S dT + \mu dN\ \text(sub\ (\ref{eq:three})\ into 
\ (\ref{eq:two}))\\
  \text{Math: } dG = \left ( \frac{\partial G }{\partial P} \right ) dP +
\left ( \frac{\partial G }{\partial T} \right ) dT +
\left ( \frac{\partial G }{\partial N} \right ) dN 
  \end{gather}
  \end{subequations}

  \begin{gather}
\text{As before, line up the terms and read off: } \nonumber\\
 \left ( \frac{\partial G }{\partial P}   \right )_{T,N} = V \\
 \left (  \frac{ \partial G}{ \partial T}  \right )_{P,N} = -S\\
 \left (  \frac{ \partial G}{ \partial N}  \right )_{T,P} = \mu \label{eq:gibbsn}
\end{gather}

\end{frame}

\begin{frame}
  \frametitle{Diffusive equilibrium and Gibbs free energy}

  \begin{itemize}
  \item Note that (\ref{eq:gibbsn}) says that the chemical potential $\mu$
for a single species  is the Gibbs free energy per molecule.


\item Finally, look at two compartments separated by a membrane
that allows thermal, mechanical and diffusive contact.  From
the results above, we know that to maximize the entropy the
systems will adjust to the same temperate and pressure.  At constant
temperature and pressure, we have as before:

\begin{gather}
dS_T = \text{0 at maximum so: }\nonumber\\
0 =  \left ( \frac{\partial S_A }{\partial N_A} \right ) dN_A + 
 \left ( \frac{\partial S_B }{\partial N_B} \right ) dN_B\nonumber\\
\text{or using (\ref{eq:diff3}): and $dN_A = -dN_B$}\nonumber\\
-\frac{\mu_A }{T}  = - \frac{\mu_B }{T} \label{eq:mufinal}
\end{gather}
 \end{itemize}

\end{frame}

\begin{frame}
  \frametitle{Diffusive equilibrium and Gibbs free energy II}

Now put this all together:
\begin{itemize}
\item From (\ref{eq:press}), (\ref{eq:thetemp}), and (\ref{eq:mufinal}) 
we know that across the liquid/gas
interface T, p, and $\mu$ are all equal in vapor and liquid
\item Since $\mu$ is the Gibbs free energy per molecule, if
  \begin{gather*}
    \mu_v = \mu_l\\
\text{then multiply by the $N_A M_w$, where $N_A$ is Avogadro's number}\\
\text{and $M_w$ is the molecular weight of water to get}\\
g_l = g_v
  \end{gather*}


   \item \textit{textbf{So two systems in diffusive, mechanical and
thermal  contact have the same Gibbs free energy/kg}}
  \end{itemize}

\end{frame}

\begin{frame}
  \frametitle{Summary}

\begin{itemize}
\item Liquid and vapor are at the same temperature because otherwise the
system could increase its entropy by moving energy from 
the hotter to colder component.
\item Liquid and vapor are at the same pressure because otherwise
the system could increase its entropy by expanding the component at higher
pressure  and shrinking the component at lower pressure
\item Liquid and vapor are at the same g because otherwise the
system could increase its entropy by moving mass from the higher g to
the lower g component
\item This last point means that $g_v = g_l$ in equilibrium.  Since
$g_v$ and $g_l$ are both only functions of temperature and pressure,
the volume of the container doesn't enter into the problem, as long
as there is enough water to maintain the equilibrium vapor pressure.
\end{itemize}


\end{frame}

\end{document}


%%% Local Variables:
%%% mode: latex
%%% TeX-master: t
%%% End:
