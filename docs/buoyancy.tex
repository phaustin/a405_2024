\documentclass[12pt]{article}
\usepackage{geometry,fancyhdr,xr,hyperref,ifpdf,amsmath,rcs,indentfirst}
\usepackage{lastpage,longtable,Ventry,url,paunits,shortcuts,smallsec,color,tightlist,float}
\geometry{letterpaper,top=50pt,hmargin={20mm,20mm},headheight=15pt} 
\usepackage[stable]{footmisc}

\pagestyle{fancy} 

\RCS $Revision: 1.5 $
\RCS $Date: 2002/01/09 03:50:54 $

\fancypagestyle{first}{
\lhead{Buoyancy and Taylor's series}
\chead{}
\rhead{page~\thepage/\pageref{LastPage}}
\lfoot{} 
\cfoot{} 
\rfoot{}
}

\ifpdf
    \usepackage[pdftex]{graphicx} 
    \usepackage{hyperref}
    \pdfcompresslevel=0
    \DeclareGraphicsExtensions{.pdf,.jpg,.mps,.png}
\else
    \usepackage{hyperref}
    \usepackage[dvips]{graphicx}
    \DeclareGraphicsRule{.eps.gz}{eps}{.eps.bb}{`gzip -d #1}
    \DeclareGraphicsExtensions{.eps,.eps.gz}
\fi

\newcommand{\dbar}{d\mkern-6mu\mathchar'26} 
\begin{document}
\newcommand{\vect}[1]{\boldsymbol{\vec{#1}}}
\pagestyle{first}


\section{Bouyancy}
\label{sec:bouyancy}




\begin{itemize}
\item this is a slightly more detailed derivation of Thompkins 1.11.1
\item Here's a short review of Taylor's series if you need a refresher: \href{https://blogs.ubc.ca/infiniteseriesmodule/units/unit-3-power-series/taylor-series/the-taylor-series}{https://blogs.ubc.ca/infiniteseriesmodule/units/unit-3-power-series/taylor-series/the-taylor-series}
\end{itemize}



One problem with the derivation of buoyancy on in most intro texts is
that it relies on the idea of a \textit{coherent air parcel}, which is a
little artificial in a turbulent atmosphere.  How do I know how big
my ``container'' is, so I can calculate the ``ambient air density'' $\rho$
and the ``parcel density'' $\rho^\prime$ in the equation

\begin{equation}
  \label{eq:container}
  F = \left (\rho - \rho^\prime \right ) g
\end{equation}
if the air isn't in a balloon? Here's another way to get to the same place
without invoking the idea of some kind of package of air.  This definition of buoyancy is
used in the 3-dimensional cloud model we will work with in class.

Start with Newton's second law:
\begin{equation}
  \label{eq:newton}
F= ma
\end{equation}
Rewrite this in terms of the vertical velocity derivative for the acceleration, 
and the pressure gradient force and gravity for the force per kg:

\begin{equation}
  \label{eq:newton2}
 - \frac{\partial p}{\partial z} - \rho g = \rho \frac{dw}{dt}
\end{equation}


Now define a horizontally uniform base state with pressure $\overline{p}$ and density
$\overline{\rho}$ that satisfy hydrostatic balance:

\begin{equation}
  \label{eq:hydro}
  d \overline{p} = \overline{\rho} g dz
\end{equation}
or equivalently, since there is no $x$ or $y$ dependence (uniform in horizontal),
we can substitute a partial for the total derivitive.

\begin{equation}
  \label{eq:hydro2}
- \frac{1}{\overline{\rho}} \frac{\partial \overline{p}}{\partial z} -  g  = 0
\end{equation}



Without approximation we can write the pressure and density as the sum
of a perturbation and these base state values:  $p=\overline{p} +
p^\prime$, $\rho=\overline{\rho} + \rho^\prime$.  
Stick these $p,\rho$ values into
(\ref{eq:newton2}):

  \begin{equation}
  \label{eq:newton3}
 \frac{dw}{dt} = -\frac{1}{(\overline{\rho} + \rho^\prime)}
 \frac{\partial }{\partial z}
(\overline{p} + p^\prime) -  g
\end{equation}

Now expand the density term in a Taylor series (Thompkins eq. 1.57):


\begin{equation}
  \label{eq:taylor}
  \frac{1}{(\overline{\rho} + \rho^\prime)} = 
\frac{1}{(\overline{\rho})} 
  \left ( \frac{1}{1 + \rho^\prime/\overline{\rho}} \right ) =
\frac{1}{(\overline{\rho})} \left [ 1 -
  \frac{\rho^\prime}{\overline{\rho}} + 
\left ( \frac{\rho^\prime}{\overline{\rho}} \right )^2 + \ldots \right  ]
\end{equation}

If we drop all but first order terms in (\ref{eq:taylor})  and plug it
into (\ref{eq:newton3}), again dropping  squared term
$p^\prime p^\prime$ we get:

\begin{equation}
  \label{eq:big}
\frac{dw}{dt} = -\frac{1}{\overline{\rho}} \frac{\partial
  \overline{p}}{\partial z}
-g - \frac{1}{\overline{\rho}} \frac{\partial p^\prime}{\partial z}  
+ \frac{1}{\overline{\rho}} \frac{\partial \overline{p}}{\partial z} 
\left ( \frac{\rho^\prime}{\overline{\rho}} \right )
\end{equation} 


  Now I can use (\ref{eq:hydro2}) twice on (\ref{eq:big}): once to
cancel the first two terms on the rhs, and once to replace the final density
term with $g$. This gives:

\begin{equation}
  \label{eq:newton4}
 \frac{dw}{dt} = -\frac{1}{\overline{\rho}} \frac{\partial p^\prime}{\partial z}
- g \left ( \frac{\rho^\prime}{\overline{\rho}} \right )
\end{equation}
The first term on the rhs is the non-hydrostatic pressure gradient acceleration, which can
be neglected under some conditions.  The 
last term on the rhs of (\ref{eq:newton4}) is the buoyancy $B$:

\begin{equation}
  \label{eq:buoyancy}
  B = - g \left ( \frac{\rho^\prime}{\overline{\rho}} \right )
\end{equation}


  As I show in class, if we differentiate the equation of state:

\begin{equation}
  \label{eq:state}
  p=\rho R_d T
\end{equation}
about the hydrostatic reference state, we get:

\begin{equation}
  \label{eq:state2}
  \frac{\rho^\prime}{\overline{\rho}}  = \frac{p^\prime}{\overline{p}}
  - \frac{T^\prime}{\overline{T}}
 \end{equation}


If we neglect the pressure perturbation in (\ref{eq:newton4}) then
we should drop it from  (\ref{eq:state2}) also and 
we get:

\begin{equation}
  \label{eq:state3}
  \frac{\rho^\prime}{\overline{\rho}}  \approx
  - \frac{T^\prime}{\overline{T}}
 \end{equation}
and

\begin{equation}
  \label{eq:buoyancy2}
  B \approx  g \left ( \frac{T^\prime}{\overline{T}} \right )
\end{equation}
which is the same as Thompkins eq. 1.61.  Note that they use $T$ for the
temperature of the environment, which I have called $\overline{T}$ here.

So what does say about the spatial extent of the buoyancy perturbation?  Suppose we
have a model that splits the atmosphere up in 25 meter x 25 meter x 25 meter cubes
over a domain that is 1 km x 1 km x 1 km.  The to calculate $\overline{\rho}$ at
any height just average over the 40 x 40 grid at that level.  To calculate 
$\overline{\rho^\prime}$ for each cell at a every height just do:

\begin{equation}
  \label{eq:rhop}
  \rho^\prime(x,y,z) = \rho(x,y,z) - \overline{\rho}(z)
\end{equation}
So the ``containers'' are the 25 m x 25 m x 25 m cubes.  Question:  would I get a different answer
if I enlarged the cubes to 50 m x 50 m x 50 m ?


\end{document}



%%% Local Variables:
%%% mode: latex
%%% TeX-master: t
%%% End:
