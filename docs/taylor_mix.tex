\documentclass[12pt]{article}
\usepackage{geometry,fancyhdr,xr,hyperref,ifpdf,amsmath,rcs,indentfirst}
\usepackage{lastpage,longtable,Ventry,url,paunits,shortcuts,smallsec,color,tightlist,float}
\geometry{letterpaper,top=50pt,hmargin={20mm,20mm},headheight=15pt} 
\usepackage[stable]{footmisc}

\pagestyle{fancy} 

\RCS $Revision: 1.5 $
\RCS $Date: 2002/01/09 03:50:54 $

\fancypagestyle{first}{
\lhead{Mixing entropies}
\chead{}
\rhead{page~\thepage/\pageref{LastPage}}
\lfoot{} 
\cfoot{} 
\rfoot{}
}

\ifpdf
    \usepackage[pdftex]{graphicx} 
    \usepackage{hyperref}
    \pdfcompresslevel=0
    \DeclareGraphicsExtensions{.pdf,.jpg,.mps,.png}
\else
    \usepackage{hyperref}
    \usepackage[dvips]{graphicx}
    \DeclareGraphicsRule{.eps.gz}{eps}{.eps.bb}{`gzip -d #1}
    \DeclareGraphicsExtensions{.eps,.eps.gz}
\fi


\begin{document}
\newcommand{\vect}[1]{\boldsymbol{\vec{#1}}}
\pagestyle{first}

On p. 57, Thompkins shows that the moist static energy mixies linearly.
Show that that is also roughly true for $\theta_e$ by doing the following:


\begin{enumerate}
\item Recall that, if we neglect the increase in entropy due to irreversible mixing,
that the entropies of two components will sum to give the entropy of the mixture:

\begin{equation}
  \label{eq:entrop}
  s_{mix} = f s_{env}  + (1-f) s_{cld}
\end{equation}
where $f$ and $(1-f)$ are the mass fractions of the environment and cloudy air and $s_{env}$ and $s_{cld}$ are the specific entropies
of each component in J/kg/K.

We also know that, if we neglect the effect of liquid and vapor on the heat capacities:

\begin{equation}
  \label{eq:S2}
  s = c_p \log \theta_e
\end{equation}
so we can rewrite (\ref{eq:entrop}) as

\begin{equation}
  \label{eq:s3}
 \log \theta_{mix} \approx f \log \theta_{env} + (1 - f) \log \theta_{cld}
\end{equation}
or equivalently, taking $\exp$ of each side:

\begin{equation}
  \label{eq:s4}
 \theta_{emix} = \theta_{eenv}^f \theta_{ecld}^{1-f}
\end{equation}

Use a Taylor series expansion to show that if $(\theta_{ecld} - \theta_{eenv})/\theta_{eenv} = \delta \ll 1$ 
then (\ref{eq:s4}) is approximately equivalent to:

\begin{equation}
  \label{eq:s5}
 \theta_{emix} \approx f \theta_{eenv} + (1 -f) \theta_{ecld}
\end{equation}


\end{enumerate}

Hint:  you can rewrite $\theta_{ecld}/\theta_{env}$ as $(\theta_{eenv} + (\theta_{ecld} - \theta_{env}))/\theta_{env}$ and
show by expanding in a Taylor series that:

\begin{equation}
  \label{eq:taylor}
  (1 + \delta)^f \approx 1 + f \delta
\end{equation}


and

\begin{equation}
  \label{eq:taylor2}
  f + (1-f) = 1
\end{equation}


\end{document}


%%% Local Variables:
%%% mode: latex
%%% TeX-master: t
%%% End:
