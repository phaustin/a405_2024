\documentclass[12pt]{article}
\usepackage{geometry,fancyhdr,xr,hyperref,ifpdf,amsmath,rcs,shortcuts}
\usepackage{lastpage,longtable,color,paunits,smallsec}
\geometry{letterpaper,top=50pt,hmargin={20mm,20mm},headheight=15pt} 


\pagestyle{fancy} 

\RCS $Revision: 1.1 $
\RCS $Date: 2006/01/15 23:38:01 $

\fancypagestyle{first}{
\lhead{A405: Entropy -- notes on Thompkins Sec. 1.9}
\chead{\textbf{}}
\rhead{page~\thepage/\pageref{LastPage}}
\lfoot{} 
\cfoot{} 
\rfoot{}
}

\ifpdf
    \usepackage[pdftex]{graphicx} 
    \usepackage{hyperref}
    \pdfcompresslevel=0
    \DeclareGraphicsExtensions{.pdf,.jpg,.mps,.png}
\else
    \usepackage{hyperref}
    \usepackage[dvips]{graphicx}
    \DeclareGraphicsRule{.eps.gz}{eps}{.eps.bb}{`gzip -d #1}
    \DeclareGraphicsExtensions{.eps,.eps.gz}
\fi

\begin{document}
\pagestyle{first}

\begin{center}
Atsc405: Discussion of Thompkins Section 1.9\\
\end{center}


\section{2nd law}
\label{sec:2nd-law-continued}

First law in pressure coords:

\begin{equation}
  \label{eq:poten1}
  q_t dt = c_p dT - \frac{R_d T}{p} dp
\end{equation}
Instead of
setting $q_t=0$ we could also have divided (\ref{eq:poten1}) by the
temperature $T$ and gotten:

\begin{equation}
  \label{eq:poten2}
  \frac{q_t dt}{T} = c_p \frac{dT}{T} - R_d \frac{dp }{p}
\end{equation}
Note the connection between (\ref{eq:poten2}) and the second law:

\begin{equation}
  \label{eq:2nd}
  d\phi \geq \frac{q_t dt}{T}
\end{equation}
where the $=$ sign holds if the process is reversible.

You should be able to show to show using the definition of $\theta$
that:

\begin{equation}
  \label{eq:diffTheta}
  c_p \frac{d\theta}{\theta} = c_p \frac{dT}{T} - R_d \frac{ dp }{p}
\end{equation}

Putting (\ref{eq:diffTheta}) together with (\ref{eq:poten2}) and (\ref{eq:2nd})
we have the atmospheric version of the 2nd law of thermodynamics for dry air:

\begin{equation}
  \label{eq:atsc2nd}
d\phi=  c_p d \log\theta = c_p \frac{dT}{T} - \frac{R_d }{p} dp \geq \frac{q_t dt}{T}
\end{equation}
or integrating between two states 0 and 1:

\begin{equation}
  \label{eq:atsc3rd}
\phi_1 - \phi_0=  c_p  \log \frac{\theta_1}{\theta_0}
\end{equation}
In other words, (\ref{eq:atsc2nd}) is saying that \textit{the potential temperature 
$\theta$ is just another name for the entropy of dry air}.  It's important to note that
I can do this integration even if the physical process linking state 0 to state 1
was irreversible (why?). Another way of saying this is to note that the rhs of (\ref{eq:poten2}) is always
equal to $d\phi$, while the lhs=$d\phi$ only for reversible processes.  We'll come
back to this point later in the course.



\end{document}
