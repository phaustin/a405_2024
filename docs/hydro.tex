\documentclass[12pt]{article}
\usepackage{geometry,fancyhdr,xr,hyperref,ifpdf,amsmath,rcs,indentfirst}
\usepackage{lastpage,longtable,Ventry,url,paunits,shortcuts,smallsec,color,tightlist,float}
\geometry{letterpaper,top=50pt,hmargin={20mm,20mm},headheight=15pt} 
\usepackage[stable]{footmisc}
\pagestyle{fancy} 

\RCS $Revision: 1.5 $
\RCS $Date: 2002/01/09 03:50:54 $

\fancypagestyle{first}{
\lhead{Hydrostatic balance and weighted averages}
\chead{}
\rhead{page~\thepage/\pageref{LastPage}}
\lfoot{} 
\cfoot{} 
\rfoot{}
}

\ifpdf
    \usepackage[pdftex]{graphicx} 
    \usepackage{hyperref}
    \pdfcompresslevel=0
    \DeclareGraphicsExtensions{.pdf,.jpg,.mps,.png}
\else
    \usepackage{hyperref}
    \usepackage[dvips]{graphicx}
    \DeclareGraphicsRule{.eps.gz}{eps}{.eps.bb}{`gzip -d #1}
    \DeclareGraphicsExtensions{.eps,.eps.gz}
\fi


\begin{document}
\newcommand{\vect}[1]{\boldsymbol{\vec{#1}}}
\pagestyle{first}

\section{Weighted Averages}
\label{sec:weighted-averages}

\subsection{Discrete case}
\label{sec:discrete-case}


Suppose you have 15 measurements of the vapor mixing ratio and want to calculate the average.  

\begin{center}
\begin{tabular}{rr}
Number of samples n & mixing ratio wv (g/kg)\\
\hline
2 & 0.17\\
5 & 0.008\\
8 & 0.022\\
\end{tabular}
\end{center}

The average vapor mixing ratio for these 15 samples is:

\begin{equation}
  \label{eq:discrete}
  \overline{w_v} = \frac{2*0.17 + 5*0.08 + 8*0.022}{2 + 5 + 8}
\end{equation}
which is exactly what you would get if you added the 15 measurements individually and divided by 15.

We can rewrite \eqref{eq:discrete} in terms of fractions as:

\begin{equation}
  \label{eq:discrete}
  \overline{w_v} = \frac{2}{15}*0.17 + \frac{5}{15}*0.08 + \frac{8}{15}*0.022 = \sum_{i=1}^{3} p_i w_{vi}
\end{equation}
and interpret $p_i$ as the probability that the sample will have vapor mixing ratio $w_{vi}$

\subsection{Continuous case}
\label{sec:continuous-case}

We can extend this definition to functions by defining the probability density $p(x)$ for variable x as follows:

\begin{quote}
  \textbf{The probability that the variable $x$ will take on a value $\hat{x}$ that lies between $x$ and $x + dx$ is
$p(x)dx$.} We also require that $0 \le p(x) \le 1$ and $\int_0^\infty p(x) dx = 1$
\end{quote}

With this definition, we can define the mean value of x in the same way as for the weighted average:

\begin{equation}
  \label{eq:mean}
  \overline{x} = \int_0^\infty x p(x) dx
\end{equation}

A couple common pdfs:

\begin{itemize}

\item \textbf{Uniform distribution:}


\begin{equation}
p(x) = 
  \begin{cases}
                  \frac{1}{b - a} & \text{for } x \in [a,b]  \\
                  0               & \text{otherwise}
\end{cases}
\end{equation}

\begin{equation}
  \label{eq:unif}
  \overline{x} = \frac{b-a}{2}
\end{equation}
(you should show that this follows from \eqref{eq:mean})


\item \textbf{Exponential distribution}

  \begin{equation}
    \label{eq:expon}
    p(x) = \lambda \exp ( -\lambda x)
  \end{equation}

  \begin{equation}
    \label{eq:exmean}
    \overline{x} = 1/\lambda
  \end{equation}
(again, show this follow from \eqref{eq:mean} using 

\href{https://en.wikipedia.org/wiki/List_of_integrals_of_exponential_functions}%
{\url{https://en.wikipedia.org/wiki/List_of_integrals_of_exponential_functions}})

\item \textbf{Gaussian distribution}

\begin{equation}
  \label{eq:gaussian}
   p(x;\mu,\sigma^2) = \frac{1}{\sigma\sqrt{2\pi}} e^{ -\frac{1}{2}\left(\frac{x-\mu}{\sigma}\right)^2 }
\end{equation}

\begin{equation}
\overline{x} = \mu
  \label{eq:gaussmean}
\end{equation}

\end{itemize}


\section{Hydrostatic balance}


For the the large scale average atmosphere, it is a very good approximation to
assume \textit{hydrostatic balance}, which is the statement that 
the vertical pressure differential across a layer provides exactly the
amount of force necessary to balance gravity:

  \begin{figure}[H]
    \begin{center}
       \input hydro.pstex_t
      \caption{hydrostatic balance for a $1 \times dz\  \un{m^3}$ layer}
      \label{fig:hydro}
    \end{center}
  \end{figure}

\vspace{0.1in}

In symbols, the balance shown by Figure \ref{fig:hydro} implies that:

\begin{equation}
  \label{eq:hydro}
  \frac{dp}{dt} = - \rho g \frac{dz}{dt}
\end{equation}
\textit{Question:  is this the same pressure p as the local pressure given by the equation of state: $p=\rho R_d T$?}

\section{Scale Heights}
\label{sec:scale-heights}

We can define a characteristic scale for the variation of pressure and density in a hydrostatic atmosphere. 
(Head's up -- I'm out of letters, so am using capital P for pressure and p for probability)


\begin{enumerate}
\item Do pressure first: Rewrite (\ref{eq:hydro}) using the ideal gas law:

  \begin{subequations}
  \begin{align}
  dP =& - \rho g dz = - \frac{P}{R_d T}  g dz \label{eq:one}\\
  d\ln P =& - \frac{g }{R_d T} dz \label{eq:two}\\
  \int_{P_0}^{P}\!\,d \ln P =& - \int_{0 }^{z}\!\frac{g }{R_d T} \,dz^\prime = - \int_{0 }^{z}\!\frac{1}{H} dz^\prime \label{eq:hydro1}
  \end{align}
\end{subequations}
where $H=R_d T/g$ is the pressure scale height.  Since temperature (and gravity) are changing with height,
we need to work with a vertical mean. To do that, define 

\begin{equation}
  f(z) = g/(R_d T(z)) = 1/H
\end{equation}

We want to average $f(z)$ over height between the surface
where $z=0$ and a top level $z_T$, with the pdf $p(z)$ for the uniform distribution \eqref{eq:unif}:

\begin{equation}
  \label{eq:uniz}
  p(z) = \frac{1}{z_T - 0} = \frac{1}{z_T}
\end{equation}

In other words:

\begin{equation}
  \label{eq:oneoverh}
  \overline{f(z)} = \int_0^{z_T} f(z) p(z) dz = \frac{\int_{0 }^{z_T}\!\frac{1}{H} dz^\prime  }{z_T} = \overline{ \left ( \frac{1 }{H} \right )}
\end{equation}
Again, to save letters, we'll write:

\begin{equation}
  \label{eq:simple}
  \overline{ \left ( \frac{1 }{H} \right )} = \frac{1}{\overline{H}}
\end{equation}

So that \eqref{eq:oneoverh} becomes

\begin{subequations}
  \begin{align}
  \label{eq:oneoverh2}
  \overline{f(z)} =&  \frac{\int_{0 }^{z_T}\!\frac{1}{H} dz^\prime  }{z_T} = \left ( \frac{1 }{\overline{H}} \right )\\
\text{and rearranging this gives:} \nonumber\\
\int_{0 }^{z_T}\!\frac{1}{H} dz^\prime =& \frac{z_T}{\overline{H}}
\end{align}
\end{subequations}
Inserting this into \eqref{eq:hydro1} gives:


\begin{subequations}
  \begin{align}
  \int_{P_0}^{P}\!\,d \ln P =& -\frac{z}{\overline{H}}\\
\text{which integrates to: }\nonumber\\
P(z) =& P_0 \exp \left ( -z/\overline{H} \right )
\end{align}
\end{subequations}

We can turn $P(z)$ into a probability density function $p(z)$ by normalizing it so $\int_0^\infty p(z) dz = 1$.
To do this note that:

\begin{equation}
  \label{eq:normP}
  \int_0^\infty P_0 \exp (-z/\overline{H} ) dz = \overline{H} P_0
\end{equation}
Dividing $P(z)$ by $\overline{H} P_0$ gives an exponential distribution with $\lambda = 1/\overline{H}$:

\begin{equation}
  \label{eq:pweight}
  p(z) = \frac{1}{\overline{H}} \exp(-z/\overline{H})
\end{equation}
and using \eqref{eq:exmean}:

\begin{equation}
  \label{eq:finalscale}
  \overline{H} = \int_0^\infty\, z\,\left [ \frac{1}{\overline{H}} \exp(-z/\overline{H}) \right ] dz
\end{equation}

\textbf{\textit{In other words, the pressure scale height is an average height weighted by the vertical pressure profile of the atmosphere.
}}

\item Now repeat this for density:

  \begin{align}
 \frac{dp }{dz}  =& \frac{d }{dz}  (\rho R_d T) = R_d \left ( \frac{d\rho }{dz} T 
              + \rho \frac{ dT}{dz} \right )  = - \rho g \\
\frac{d\rho }{dz}  =& -\frac{\rho }{T}  \left ( \frac{g }{R_d} + \frac{ dT}{dz} \right )\\
\frac{d\rho }{\rho} =& - \left ( \frac{1 }{H} + 
                   \frac{1 }{T} \frac{dT }{dz} \right ) dz \equiv - \frac{dz }{H_\rho} 
  \end{align}
The vertical average is taken in exactly the same way as before to get the density profile:

\begin{align}
  \ln \rho/\rho_0 =& - \frac{z }{\overline{H_\rho }} \\
  \rho =& \rho_0 \exp \left ( - \frac{z }{\overline H_\rho} \right )
\end{align}

Using  python to calculate pressure and density scale heights for a typical midlatitude summer sounding gives
me $H\approx 8$ km and $H_\rho \approx 10$ km.


\end{enumerate}



\end{document}

%%% Local Variables:
%%% mode: latex
%%% TeX-master: t
%%% End:
