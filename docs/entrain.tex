\documentclass[12pt]{article}
\usepackage{geometry,fancyhdr,xr,hyperref,ifpdf,amsmath,rcs,indentfirst}
\usepackage{lastpage,longtable,Ventry,url,paunits,shortcuts,smallsec,color,tightlist,float}
\geometry{letterpaper,top=50pt,hmargin={20mm,20mm},headheight=15pt} 
\usepackage[stable]{footmisc}

\pagestyle{fancy} 

\RCS $Revision: 1.5 $
\RCS $Date: 2002/01/09 03:50:54 $

\fancypagestyle{first}{
\lhead{Mixing}
\chead{}
\rhead{page~\thepage/\pageref{LastPage}}
\lfoot{} 
\cfoot{} 
\rfoot{}
}

\ifpdf
    \usepackage[pdftex]{graphicx} 
    \usepackage{hyperref}
    \pdfcompresslevel=0
    \DeclareGraphicsExtensions{.pdf,.jpg,.mps,.png}
\else
    \usepackage{hyperref}
    \usepackage[dvips]{graphicx}
    \DeclareGraphicsRule{.eps.gz}{eps}{.eps.bb}{`gzip -d #1}
    \DeclareGraphicsExtensions{.eps,.eps.gz}
\fi

\newcommand{\dbar}{d\mkern-6mu\mathchar'26} 
\begin{document}
\newcommand{\vect}[1]{\boldsymbol{\vec{#1}}}
\pagestyle{first}

\section{Introduction}
\label{sec:introduction}


In the real atmosphere air is constantly mixing.  How do we do
thermodynamics on this ``open system''?  On page 57, equations 3.6 to
3.9 Thompkins shows that for unsaturated air, the vapor mixing ratio
and the temperature both mix approximately linearly, which is why
you can get supersaturated vapor on a cold day and see your
breath.   But what happens when we add the fact that mixtures can
also condense and evaporate water vapor and liquid water?

  \begin{itemize}
  \item Use conservation of enthalpy to get a new version of the
    first law, assuming that mixing doesn't change the enthalpy of
    either of the mixtures, so that the mixing is ``conservative''
   (i.e. like mixing blue and red marbles, or chemicals that don't
   react with each other).

  \item Suppose we have a cloud $c$ mixing in environmental air $e$


  \item Let a ``c'' subscript denote the cloud parcel and an ``e'' subscript denote
   environment. Between time $t_0$ and $t_0 + \Delta t$ its mass grows by $\Delta m_c =
    m_{c}(t + \Delta t) - m_{c0} = m_e$  i.e., it's taken a gulp of environmental
    air with mass $m_e$.

  \item That new mass brings in environmental enthalpy $h_e$, which
    doesn't change during the mixing process.
  \item So the new enthalpy is:
    \begin{gather}
      H_{new} = (m_{c} h_c + m_e h_e) = (m - m_e)h_c + m_e h_e \nonumber \\
H_{new} = m h_c + m_e (h_e - h_c)\\
\text{where $m=m_0 + m_e$}\\
\text{rearranging: }\nonumber \\
m (h_{new} - h_c) = m \Delta h_{mixing} = m_e (h_e - h_c) \\
\text{in the limit $m_e \rightarrow dm$ and $\Delta h_{mixing}$ = $dh_{mixing}$}\nonumber \\
dh_{mixing} = \frac{ dm}{m} (h_e - h_c) \label{eq:mixh}
    \end{gather}

    \end{itemize}




How does mixing effect the second law?  The second
law still holds if the enthalpy is changing from mixing:


\begin{gather}
  T ds  =  dh_{mixing} - \alpha dp
\label{eq:tds}
\end{gather}
If we are at constant pressure then $dp=0$ and using
(\ref{eq:mixh}) and the definition of $\theta_e$ in terms of entropy (s)
we can rewrite (\ref{eq:tds}) as:

\begin{gather}
   ds = c_p \frac{d \theta_e }{\theta_e} = c_p \frac{ d\theta}{\theta}
   + d \left ( \frac{l_v r_s}{T} \right ) =  \frac{ dm}{m} \frac{ (h_e - h_c)}{T}\label{eq:simple}\\
\text{expand the environment - saturated cloud $(h_e - h_c)$: }\nonumber \\
h_e - h_c =  (c_p T_e + l_v\, r_{ve} + g z) - (c_p T_c + l_v\, r_s(T_c,p) + g z)\\
\text{rearrange to get  an express for the change }\\
  \frac{d\theta_c}{\theta_c} + d \left ( \frac{l_v\,r_s(T_c,p)}{T_c} \right ) = 
  - \left [ \frac{T_c - T_e}{T_c} + \frac{l_v}{c_p T_c} (r_s(T_c,p) - r_{ve}) \right ]
   \frac{dm}{m}
\label{eq:tds2}
\end{gather}

So given the pressure and the environmental temperature and vapor mixing ratio you could
iteratively solve \eqref{eq:tds2} for the cloud temperature $T_c$ (note that $T_c$ is also
inside $\theta_c$ and $r_s$).   This shows all the terms involving the vapor mixing ratio $r$ which
are missing when you mix dry air but contribute strongly to the temperature change for 
a cloud mixture.  Luckily there is an easier way to find the mixed parcel temperature: use \eqref{eq:simple}  to 
integrate the time evolution of the entropy parcel entropy $ds/dt$, in terms of the mixing rate $\frac{1}{m}\frac{dm}{dt}$. 
So how do we find the fraction of entrained air, $dm/m$?

The simplest assumption is  that the cloud is mixing
like an \textit{ascending thermal} with \textit{entrainment rate}
$\lambda$:

\begin{equation}
  \label{eq:plume}
  \frac{1}{m} \frac{dm}{dt} = \lambda
\end{equation}
Equation (\ref{eq:plume}) says that the bigger the parcel gets, the
more it mixes, possibly because the larger surface area provides more
of an interface to mix across.  You should convince yourself that
the solution to (\ref{eq:plume}) for constant $\lambda$ is:

\begin{equation}
  \label{eq:sol}
  m(t) = m_0 \exp ( \lambda  t)
\end{equation}

So how do we use this?  We will need to use an ode integrator in python
to solve this equation:


\begin{equation}
  \label{eq:odeent}
     \frac{ds}{dt} = \lambda \frac{ (h_e - h_c)}{T}
\end{equation}

We also need to track the height as a function of time as well as the total water in
the parcel.
If we know the total water $r_{Tc}$, we can find temperature, $r_s$,
$r_l$ as usual by iterating to find the mixture temperature $T_c$
that satisfies $c_p \log \theta_{e}(T_c, p) = c_p \log \theta_{ec} $.  Given the temperature
of the cloud and the environment, we can find the buoyant acceleration and get
the updraft speed:

  \begin{gather}
\frac{ dz}{dt} = \text{buoyant acceleration} 
  \end{gather}


The equation for $r_{Tc}$ is exactly like $s_c$:

\begin{gather}
  \frac{dr_{Tc}}{dt} = \lambda (r_{Te} - r_{Tc})
\end{gather}

To summarize,  we want to solve for $\theta_{ec}$, $r_{Tc}$ for an ascending cloud parcel given
a cloud base $\theta_e$ and $r_T$, a constant entrainment rate $\lambda$, and an environmental
sounding.

\end{document}




%%% Local Variables:
%%% mode: latex
%%% TeX-master: t
%%% End:
