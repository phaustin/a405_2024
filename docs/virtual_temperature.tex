\documentclass[12pt]{article}
\usepackage{geometry,fancyhdr,xr,hyperref,ifpdf,amsmath,rcs,indentfirst}
\usepackage{lastpage,longtable,Ventry,url,paunits,shortcuts,smallsec,color,tightlist,float}
\geometry{letterpaper,top=50pt,hmargin={20mm,20mm},headheight=15pt}
\usepackage[stable]{footmisc}

\pagestyle{fancy} 

\RCS $Revision: 1.5 $
\RCS $Date: 2002/01/09 03:50:54 $

\fancypagestyle{first}{
\chead{Virtual temperature}
\lhead{2024/1/30}
\rhead{page~\thepage/\pageref{LastPage}}
\lfoot{} 
\cfoot{} 
\rfoot{}
}

\ifpdf
    \usepackage[pdftex]{graphicx} 
    \usepackage{hyperref}
    \pdfcompresslevel=0
    \DeclareGraphicsExtensions{.pdf,.jpg,.mps,.png}
\else
    \usepackage{hyperref}
    \usepackage[dvips]{graphicx}
    \DeclareGraphicsRule{.eps.gz}{eps}{.eps.bb}{`gzip -d #1}
    \DeclareGraphicsExtensions{.eps,.eps.gz}
\fi

\begin{document}
\pagestyle{first}


\section{Total (or material) derivative}
\label{sec:total-derivitive}

Make sure you understand
the important distinction between the total and partial derivative
on page 6 of Stull Chapter 1.   We need them both, because we use the
total derivitive to write \textit{conservation laws} (for mass, momentum,
energy and entropy) in a reference frame moving with the flow, but
models and observations are generally constructed/measured in a fixed 
(eulerian) reference frame with $x,y,z$ constant. Here are two different approaches
to deriving the p.~6 expression.  
For the first derivation it helps to  to recall some basic facts about
\href{http://en.wikipedia.org/wiki/Taylor_series}{Taylor's series}
(actually Gregory's series) and
\href{http://bit.ly/TaylorsTheorem}{Taylor's theorem}

In particular in the neighborhood of the point $t=t_0$, $x=a$, $y=b$

\begin{eqnarray*}
f(x,y)
&\approx& f(a,b,t_0) + (t-t_0)\frac{\partial f }{\partial t}(a,b,t_0) +(x-a)\frac{\partial f }{\partial x}(a,b,t_0) +(y-b)\, \frac{\partial f }{\partial y}(a,b,t_0) \\
&+& \mathrm{higher\ order\ terms}
\end{eqnarray*}
or keeping only terms of first order and rearranging

\begin{eqnarray*}
f(x,y,t) - f(a,b,t_0) = \Delta f
&\approx& \Delta t \frac{\partial f }{\partial t}(a,b,t_0) +  \Delta x \frac{\partial f }{\partial x}(a,b,t_0) + \Delta y\, \frac{\partial f }{\partial y}(a,b,t_0)
\end{eqnarray*}
and dividing through by $\Delta t$:

\begin{eqnarray*}
\frac{\Delta f }{\Delta t} 
&\approx& \frac{\Delta t }{\Delta t} \frac{\partial f }{\partial t}(a,b,t_0) +  \frac{\Delta x }{\Delta t} \frac{\partial f }{\partial x}(a,b,t_0) + \frac{\Delta x }{\Delta t} \frac{\partial f }{\partial y}(a,b,t_0) \\
&=& \frac{\partial f }{\partial t} +  \boldmath{u} \cdot \nabla f
\end{eqnarray*}

But how do we know that $\frac{\Delta f }{\Delta t}$ is the \textit{material} derivative -- i.e. the
derivative in a reference frame moving with the flow?  Take a look 
at two fluid mechanics lectures by
\href{http://star-www.st-and.ac.uk/~mmj}{Moira Jardine} of the University
of St.~Andrews, \href{https://www.dropbox.com/s/e2q2eruf7d0kk4w/fluids_01.pdf?dl=0}{Fluid lecture 1} and
\href{https://www.dropbox.com/s/7d89qa4v2kus0r6/fluids_02.pdf?dl=0}{Fluid lecture 2}.

Her definition of $\frac{ dQ}{dt}$ is:


\begin{equation}
  \label{eq:dqdt}
  \frac{ dQ}{dt} = \frac{ Q(r + \delta r, t + \delta t) - Q(r,t) }{\delta t}
\end{equation}
with the implicit idea that $Q(r + \delta r, t + \delta t)$ is the same piece of fluid
as $Q(r,t) {\delta t}$, observed at time $t + \delta t$ -- that is, the displacement
$\delta r$ is whatever is needed to track the fluid parcel.

If this seems a little ad hoc,
it is possible to make this more
explicit by specifically labeling the fluid volumes by their
initial positions, and being clear about the fact that we
are following the volumes as time progresses.  Here I'm
relying on a derivation 
given in Salmon, 1998, Lectures on Geophysical Fluid Dynamics.

In Salmon's notation, for the
\textit{Eulerian description}, the independent variables are the space
coordinates $\mathbf{x}=(x,y,z)$ and the time $t$.  In the
\textit{Lagrangian description}, the independent coordinates are a
set of particle labels which uniquely identify a small region of the
fluid.  For example the label could be the three coordinates
$\mathbf{a}=(a,b,c)$, where $(a,b,c)$ have the numerical values
of $(x,y,z)$ at time $t=0$.  It is also helpful to give a separate
symbol to the time in the Lagrangian frame, $\tau$, to remind us that
as $\tau$ varies the label $(a,b,c)$ is kept fixed, while as $t$, varies
the position $(x,y,z)$ is kept fixed.  In this notation the material
or total derivative is by definition the time derivitive keeping the
labels constant:

\begin{equation}
  \label{eq:material}
  \frac{\partial Q}{\partial \tau} = \frac{d Q}{dt}
\end{equation}
where $Q$ is a property of the fluid, like vapor mixing ratio
or energy. Eq. \eqref{eq:material} gives  the rate of change of $Q$ for a tagged region of 
of fluid, with evolving spatial coordinates given by $x(a,b,c,\tau)$,
$y(a,b,c,\tau)$, $z(a,b,c,\tau)$.

Then using the chain rule we have:

\begin{equation}
  \label{eq:chain}
  \frac{\partial Q}{\partial \tau} = \frac{d Q}{dt} = 
\frac{\partial Q}{\partial t} \frac{\partial t}{\partial \tau} + 
\frac{\partial Q}{\partial x} \frac{\partial x}{\partial \tau} +
\frac{\partial Q}{\partial y} \frac{\partial y}{\partial \tau} +
\frac{\partial Q}{\partial z} \frac{\partial z}{\partial \tau}
\end{equation}
But the velocity of the infinitesimal volume is by definition:

\begin{equation}
  \label{eq:velocity}
  \mathbf{v} \equiv 
\left ( \frac{\partial x}{\partial \tau} , \frac{\partial y}{\partial \tau}, 
   \frac{\partial z}{\partial \tau} \right ) =
\left ( u, v, w \right )
\end{equation}
So we can rewrite (\ref{eq:chain}) as:

\begin{equation}
  \label{eq:chain2}
  \frac{\partial Q}{\partial \tau} = \frac{d Q}{dt} = 
\frac{\partial Q}{\partial t} +
u \frac{\partial Q}{\partial x}  + 
v \frac{\partial Q}{\partial y} +
w \frac{\partial Q}{\partial y} =
\frac{\partial Q}{\partial t} +
\mathbf{v} \cdot \nabla Q
\end{equation}
Thus   $\mathbf{v} \equiv 
   \left ( \frac{\partial x}{\partial \tau} , \frac{\partial y}{\partial \tau}, 
   \frac{\partial z}{\partial \tau} \right )$
makes explicit the idea that $\frac{\delta r }{\delta t}$ is the velocity
``following the flow''.


Then Taylor's frozen turbulence hypothesis is equivalent to the statement that:

\begin{equation}
  \label{eq:taylor}
\frac{\partial Q}{\partial \tau} =  \frac{d Q}{dt} = 0
\end{equation}
i.e., there are no sources or sinks of $Q$ changing the properties of the
fluid in the infinitesimal box.    In particular, the turbulence is
\textit{frozen}, and is not able to mix adjacent boxes on the time and
space scales of interest in the problem.  This doesn't mean that
there are no fluctuations, it just means that the \textit{statistics} of 
the fluctuations are stationary.  (Note that
the idea of frozen turbulence is due to 
\href{http://en.wikipedia.org/wiki/Geoffrey_Ingram_Taylor}{G. I. Taylor} (b.~1886, d.~1075)
not \href{http://en.wikipedia.org/wiki/Brook_Taylor}{Brook Taylor} (b.~1685, d.~1731). 


\section{Virtual temperature}
\label{sec:virtual-temperature}

A nice example of the use of Taylor's series is Stull (1.5.1a) on p.~7:

\begin{equation*}
  \theta_v = \theta ( 1 + 0.61 r_{sat} - r_l )
\end{equation*}
where $\theta$ called the \textit{potential temperature} is acutally a measure
of the entropy of dry air (to be derived later), $r_{sat}$ is the saturation
mixing ratio for water vapor (kg water/kg dry air) and $r_l$ is the
mixing ratio for cloud droplets (kg water/kg dry air).

We'll go into all of this in more detail, but it's worth looking at how
the factor 0.61 comes about in equations for density like 
\href{https://www.eoas.ubc.ca/books/Practical_Meteorology/prmet102/Ch03-thermo-v102b.pdf}%
{Stull PM Chapter 3 eq. 3.15)}
Virtual temperature provides a succinct way to describe the density of a mixture of
dry air, water vapor, and hydrometeors like cloud droplets, raindrops, snow and ice.
Specifically, write this density as:

  \begin{equation}
    \label{eq:dalton}
    \rho = \rho_d + \frac{e}{R_v\, T} + \rho_l + \rho_r + \rho_i
  \end{equation}
where $\rho_d$ is the density of dry air, $e$ is the vapour pressure, $R_v$ is the
gas constant for water vapor (461 \jkgk), $T$ is the temperature and
$\rho_l,\  \rho_r,\ \rho_i$ are the densities of cloud droplets, rain drops and ice
crystals.  (When we begin comparing densities to find the \textit{buoyancy} we'll
need the additional assumption that the droplets, drops and crystals are all
falling at constant velocity, so we can calculate the downward force they exert
on the surrounding air). From the definition of the mixing ratio we know that:

\begin{equation}
  \label{eq:mixing}
  r_v = \frac{\rho_v}{\rho_d} = \frac{\frac{e}{R_v T}}{\frac{p_d}{R_d T}} = \frac{R_d}{R_v} \frac{e}{p-e}
      = \epsilon \frac{e}{p-e} \approx 0.622 \frac{e}{p-e}
\end{equation}

Inverting (\ref{eq:mixing}) gives:

\begin{equation}
  \label{eq:invert}
  \frac{e}{p} = \frac{r_v}{r_v + \epsilon}
\end{equation}

Putting in the equation of state for dry air and group terms, (\ref{eq:dalton}) becomes:

\begin{equation}
  \label{eq:rho2}
      \rho = \frac{p}{R_d T} \left ( 1 - \frac{e}{p} (1 - \epsilon) \right ) + \rho_l + \rho_r + \rho_i
\end{equation}
and using (\ref{eq:invert}):

\begin{equation}
  \label{eq:rho3}
      \rho = \frac{p}{R_d T} \left ( 1 - \frac{r_v}{r_v + \epsilon} (1 - \epsilon) \right ) + \rho_l + \rho_r + \rho_i
\end{equation}

Now divide both sides of (\ref{eq:rho3}) by $\rho_d = (p-e)/(R_d T)$

\begin{equation}
  \label{eq:rho4}
  \frac{\rho}{\rho_d} = \left ( \frac{p}{R_d T_v} \frac{R_d T}{p -e} \right ) =
\frac{p}{R_d T} \frac{R_d T}{p-e} \left [ 1 -  \frac{r_v}{r_v + \epsilon} (1 - \epsilon) \right ] + r_l + r_r  + r_i
\end{equation}
where we've defined the virtual temperature, $T_v$ as the temperature that produces the correct
density $\rho$ for the mixture given the (incorrect) dry air gas constant $R_d$.

Cleaning this up by moving multiplying by $(p-e)/p$:

\begin{equation}
  \label{eq:rho5}
 \frac{T}{ T_v} = 
 \left [ 1 -  \frac{r_v}{r_v + \epsilon} (1 - \epsilon) \right ] +  \frac{p-e}{p} \left [ r_l + r_r  + r_i \right ] 
\end{equation}

But we know that in the atmosphere, $e/p,\ r_l,\ r_r,\ r_i$ are all small (below 0.02) so 
neglect their products, which leaves:

\begin{equation}
  \label{eq:rho6}
 \frac{T}{ T_v} = 
 \left [ 1 -  \frac{r_v}{r_v + \epsilon} (1 - \epsilon) \right ] +   r_l + r_r  + r_i 
\end{equation}
 
Now rearrange (\ref{eq:rho6}), dropping all second order terms:

\begin{equation}
  \label{eq:rho7}
 \frac{T}{T_v} = \frac{r_v + \epsilon - r_v + r_v \epsilon + \epsilon (r_l + r_r + r_i) }{r_v + \epsilon}
\end{equation}

or flipping:

\begin{equation}
  \label{eq:rho8}
 T_v =  T \left ( \frac{r_v + \epsilon}{\epsilon} \right ) \left ( \frac{ 1  }{1 + (r_l + r_r + r_i)}
 \right )
\end{equation}

To get the mixing ratios out of the denominator, use a Taylor series expansion:

\begin{equation}
  \label{eq:taylor}
  f(x)  = f(x_0) + f^\prime(x_0)(x - x_0) + \frac{f^{\prime\prime}(x_0)}{2}(x-x_0)^2 +
            \frac{f^{\prime\prime\prime}(x_0)}{2\cdot3}(x-x_0)^3 + \ldots
\end{equation}

You should show that expanding about $x_0 = 0$ yields:

\begin{equation}
  \label{eq:expan}
\frac{1}{1 + r} = 1 - r + r^2 - r^3 + \ldots  
\end{equation}

So that


\begin{equation}
  \label{eq:rho9}
 T_v \approx  T \left ( \frac{r_v + \epsilon}{\epsilon} \right ) \left ( 1 - (r_l + r_r + r_i)
 \right )
\end{equation}

and rearranging and again dropping second order terms:

\begin{equation}
  \label{eq:rho10}
 T_v \approx   \frac{T (\epsilon + (1-\epsilon) r_v - \epsilon(r_l + r_r + r_i))}{\epsilon}
\end{equation}

and finally:

\begin{equation}
  \label{eq:rho11}
 T_v \approx   T \left ( 1 + \frac{( 1-\epsilon)}{\epsilon} r_v - (r_l + r_r + r_i) \right )
 = T \left ( 1 + 0.608\, r_v - (r_l + r_r + r_i) \right )
\end{equation}

In words \eqref{eq:rho11} says that air becomes more buoyant (effectively ``warmer'')
as $r_v$ increases (because light $H_2O$ replaces heavy $N_2$), and droplets,
drops and ice concentrations decrease (because falling hydrometeors exert downward
drag on the
atmosphere).


\end{document}
%%% Local Variables:
%%% mode: latex
%%% TeX-master: t
%%% End:
