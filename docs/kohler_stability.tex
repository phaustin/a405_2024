\documentclass[12pt]{article}
\usepackage{xr,geometry,fancyhdr,hyperref,ifpdf,amsmath,rcs}
\usepackage{color,lastpage,longtable,Ventry,url,paunits,shortcuts,smallsec,float}
\geometry{letterpaper,top=50pt,hmargin={20mm,20mm},headheight=15pt} 

\pagestyle{fancy} 


\RCS $Revision: 1.1 $
\RCS $Date: 2006/03/08 17:38:15 $

\fancypagestyle{first}{
\lhead{\textbf{A405}}
\chead{\textbf{Kohler stability}} 
\rhead{p.~\thepage/\pageref{LastPage}}
\lfoot{} 
\cfoot{} 
\rfoot{}
}

\ifpdf
    \usepackage[pdftex]{graphicx} 
    \usepackage{hyperref}
    \pdfcompresslevel=0
    \DeclareGraphicsExtensions{.pdf,.jpg,.mps,.png}
\else
    \usepackage{hyperref}
    \usepackage[dvips]{graphicx}
    \DeclareGraphicsRule{.eps.gz}{eps}{.eps.bb}{`gzip -d #1}
    \DeclareGraphicsExtensions{.eps,.eps.gz}
\fi
\externaldocument[lec1,]{lec1}
\externaldocument[lec2,]{lec2}
\externaldocument[lec3,]{lec3}
\hyperbaseurl{http://www.eos.ubc.ca/courses/atsc405/}
\begin{document}
\pagestyle{first}

\section{Stability of the Kohler equation}

Recall that we had the following result for the derivative of the thermodynamic potential $G$ (which is called
$G$ by Lohmann):


\begin{equation}
  \label{eq:dEdr}
  \frac{\delta G}{\delta r} = \left [ -R_v T \rho_l 4 \pi r^2 \ln \left ( \frac{S}{a_w}
    \right ) + 8 \pi r \sigma \right ]
\end{equation}

\noindent
The K\"ohler curve consists of those values of droplet radius $r$ and
saturation $S$ such that $G$ is at an extremum.  Setting $\delta G/\delta r$=0 in (\ref{eq:dEdr})
and rearranging gives (you should show this):

\begin{equation}
  \label{eq:kohler}
  S_{equil} = \frac{e_{\chi c}}{e_s(T)} =  a_w \exp \left [ \frac{2\sigma}{\rho_l R_v T r} \right ]
\end{equation}
and repeating this figure:


\includegraphics[width=7.5in]{kohler_fig}


In the problem set you find values for the maximum of $S_{equil}$, which occurs
at radius $r_{crit}$ and equilibrium relative humidit $S_{crit}$.


We can show explicitly that 
that haze particles with $r < r_{crit}$ arte stable
at a particular $S=e/e_s$, while cloud droplets with $r > r_{crit}$ grow
rapidily. To see this, we need to check
the sign of $\delta G^2/\delta r^2$.  To check this, first take apart the logarithm
in (\ref{eq:dEdr}) and rewrite:

\begin{equation}
  \frac{\delta G}{\delta r} = \left [ -R_v T \rho_l 4 \pi r^2 \left ( \ln S - \ln a_w
    \right ) + 8 \pi r \sigma \right ] 
  \label{eq:nouse}
\end{equation}
Now use the chain rule:

\begin{equation}
  \frac{\delta ^2G}{\delta r^2} = - 4 \pi R_v T \rho_l \left [   2 r \ln S - 2 r\ln a_w
    -\frac{r^2}{a_w} \frac{a_w}{dr} \right ]  + 8 \pi \sigma
\label{eq:expand}
\end{equation}
Note that  $a_w$ varies with radius, but S does
not, since it is given by the environment.


To get the rest of the way, you need to recognize that we want to
evaluate (\ref{eq:expand}) \textit{on the K\"ohler curve}, which means
we want S given by (\ref{eq:kohler}), so that:

\begin{equation}
  \label{eq:newkoh}
  \ln S = \ln a_w + \frac{a}{r}
\end{equation}

Plugging (\ref{eq:newkoh}) into (\ref{eq:expand}) gives:


\begin{equation}
  \label{eq:dg2dr2}
\frac{\delta ^2 G}{\delta r^2} = -4 \pi R_v T \rho_l \left ( 2 r \left ( \ln(a_w)
    + \frac{a}{r} \right ) -2 r \ln (a_w ) - \frac{r^2}{a_w}
  \frac{da_w}{dr} \right ) + 8 \pi \sigma
\end{equation}

\noindent
Where $a=\frac{2 \sigma}{\rho_l R_v T}$.


If we define   $b= \frac{i m M_w}{ 4/3\pi \rho_s M_s}$
and assume that the following approximations hold:

\begin{equation}
  \label{eq:approx}
  a_w \approx 1 - \frac{b}{r^3},\  \frac{1}{a_w} \approx 1 +
  \frac{b}{r^3},\  \frac{da_w}{dr}=\frac{3b}{r^4}
\end{equation}

\noindent
then using this approximation yields:

\begin{equation}
  \label{eq:dg2dr2c}
  \frac{\delta ^2G}{\delta r^2} = - 4 \pi R_v T \rho_l \left [ 2 a - r^2 \left ( 1 +
    \frac{b}{r^3} \right ) \frac{3b}{r^4}  \right ] + 8 \pi \sigma
\end{equation}

Neglecting the $\frac{3 b^2}{r^5}$ term because it's tiny gives:




\begin{equation}
  \label{eq:dg2dr2b}
  \frac{\delta ^2 G}{\delta r^2} \approx -4 \pi R_v T \rho_l \left ( 2a -
    \frac{3b}{r^2} \right ) + 8 \pi \sigma
\end{equation}

Finally, setting (\ref{eq:dg2dr2b})=0 
shows you that $\frac{\delta G^2}{\delta r^2}$ will change sign at $r_{crit}$ 
given by: 


\begin{equation}
  \label{eq:rcrit}
  r_{crit}  = \left ( \frac{3b}{a} \right )^\frac{1}{2}
\end{equation}
Testing $\frac{\delta ^2 G}{\delta r^2}$ with other $r$ values should show you that
(\ref{eq:dg2dr2}) changes sign (switches from a stable to unstable
equilibrium when $r > r_{crit}$.


The critical radius $r_{crit}$ 
distinguishes \textit{haze particles} from \textit{cloud droplets}.
Haze particles have $r < r_{crit}$ and are in stable equilibrium at
saturation $S$. Cloud droplets are \textit{activated} from haze
particles when they grow enough so that $r > r_{crit}$.  This is how
a thermal diffusion cloud chamber works (see figure below which is similar to 
the last figure in Lohmann Chapter 6):

 \begin{figure}[htbp]
%   \centering
  \includegraphics{chamber}
%   \caption{Cloud chamber using a thermal gradient to control $S$}
%   \label{fig:chamber}
 \end{figure}

This means that the equilibrium vapor curve given by (\ref{eq:kohler}) cannot
be used by itself to predict the vapor pressure for a population of
activated droplets with $r > r_{crit}$, because the unstable system will
continue to grow larger and larger droplets, lowering its total
G.  To track the growth of these droplets requires a more sophisticated
model which we will develop next.

The fact that aerosols start scattering much more light when they activate and grow
means that a laser, shining through the cloud chamber, can measure the scattering
as the supersaturation in the cloud chamber is increased.  The scattering will go up
as smaller and smaller aerosols are activated at higher and higher saturations.
The relationship between the critical supersaturation and the aerosol mass means
that we can get a count of $N(D)$, the number of aerosols with diameters larger than
D, and then use that to figure out $n(D)dD$, the number of aerosols in the size
range $D \leftarrow D + dD$.



\end{document}

%%% Local Variables:
%%% mode: latex
%%% TeX-master: t
%%% End:
