\documentclass[12pt]{article}
\usepackage{geometry,fancyhdr,xr,hyperref,ifpdf,amsmath,rcs,indentfirst}
\usepackage{lastpage,longtable,Ventry,url,paunits,shortcuts,smallsec,color,tightlist,float}
\geometry{letterpaper,top=50pt,hmargin={20mm,20mm},headheight=15pt} 
\usepackage[stable]{footmisc}

\pagestyle{fancy} 

\RCS $Revision: 1.5 $
\RCS $Date: 2002/01/09 03:50:54 $

\fancypagestyle{first}{
\lhead{Temperature and the kinetic theory of gasses}
\chead{}
\rhead{page~\thepage/\pageref{LastPage}}
\lfoot{} 
\cfoot{} 
\rfoot{}
}

\ifpdf
    \usepackage[pdftex]{graphicx} 
    \usepackage{hyperref}
    \pdfcompresslevel=0
    \DeclareGraphicsExtensions{.pdf,.jpg,.mps,.png}
\else
    \usepackage{hyperref}
    \usepackage[dvips]{graphicx}
    \DeclareGraphicsRule{.eps.gz}{eps}{.eps.bb}{`gzip -d #1}
    \DeclareGraphicsExtensions{.eps,.eps.gz}
\fi


\begin{document}
\newcommand{\vect}[1]{\boldsymbol{\vec{#1}}}
\pagestyle{first}


\section{Kinetic theory of gasses}

\textit{This is a simpler version of the derivation that appears on p. 188 of the PHYS 203 text}


What is temperature?  For an ideal dilute (i.e. no intermolecular attraction) 
 monotomic gas, it turns out that
temperature is proportional to the kinetic energy of the atoms.  To see this
start with  the ideal gas law (taken as experimental fact):

  \begin{equation}
    \label{eq:state}
    pV= nR^* T
  \end{equation}
where $p$ is the pressure, $V$ the volume, $n$ the number of moles, $R^*$ the
universal gas constant (8314 \un{J\,Kmole^{-1}\,K^{-1}}).

But where does (\ref{eq:state}) come from?

Suppose we have a
container of length $L$, area $A$ and therefore volume $V=AL$, filled
with several moles of hydrogen atoms traveling with velocity
$\vec{v}=v_x \vec{i} + v_y \vec{j} + v_z \vec{k}$.

\par 

If the volume is large enough so that the atoms occupy a negligible fraction 
(\textbf{why this restriction?}), 
then
in the x dimension for a single molecule Newton's second law gives:
\begin{subequations}
  \label{eq:second}
\begin{eqnarray}
  F_x &=& \frac{d((mv)_x)}{dt} \approx \frac{\Delta((mv)_x)}{\Delta t}\label{eq:seconda}\\
  \Delta (mv)_x &=& mv_x - m(-v_x) = 2mv_x \label{eq:secondb}\\
\Delta t &=& \frac{2 L}{v_x}
\end{eqnarray}
\end{subequations}
where $m$ is the mass of a molecule. \textbf{Why is $2L$ the appropriate length here?}

Now use (\ref{eq:secondb})  to get the force in the x direction, averaged over all
molecules.  We want to do an \textit{ensemble} (or probability) average.
It can be shown that the molecular speeds $v$ follow
\textit{Maxwell's distribution}:

\begin{equation}
  \label{eq:maxwell}
  f(v) = C  v^2 \exp(-m v^2 /(2 k T))
\end{equation}
where $k$ is Boltzman's constant, $T$ is the temperature and $C=\sqrt{\frac{2}{\pi}\left(\frac{m}{kT}\right)^3}$
and $f(v)dv$ is the probability that a speed is in the range
$v \rightarrow v + dv$.

So use this to define the ensemble average speed as:

\begin{equation}
  \label{eq:ensemble}
  \left < v \right > = \int_0^\infty v f(v) dv
\end{equation}

Note that
(\ref{eq:ensemble}) is one-sided (it gives the distribution of molecular speeds,
not velocities and speed is never negative).  We assume the speed distribution
speed distribution is both \textit{isotropic} (independent of direction) and
\textit{homogeneous} (independent of position).

For one molecule of mass $m$ the average force during a collision with a wall is:

\begin{equation}
  \label{eq:avg}
    \left < F_x \right > = \left < \frac{2 m v_x}{2\frac{L}{v_x}} \right > 
                = \left < \frac{m(v_x)^2}{L} \right > 
\end{equation}

And for n moles with $N_A$ molecules per mole we get the total force and
the force/area (pressure) because we have averaged over a long enough time
for every molecule to hit the right wall exactly once.  \textbf{Question:
but what about collisions with other molecules?}:

\begin{subequations}
\begin{eqnarray}
  \label{eq:rms}
  \sum \left < F_x \right > &=& \frac{n N_A m \left < (v_x)^2 \right >}{L} \\
  p &=& \frac{\sum \left < F_x \right > }{A} = \frac{nN_A m \left < (v_x)^2 \right >}{V}\label{eq:press1}
\end{eqnarray}
\end{subequations}
You should be able to use  (\ref{eq:ensemble}) to show that:

\begin{equation}
  \label{eq:3comp}
  \left < v^2 \right > =   \left < v_x^2 + v_y^2 + v_z^2 \right > = 
    \left < v_x^2 \right >  +  \left < v_y^2 \right > + \left < v_z^2 \right > = 3 \left < v_x^2 \right >
\end{equation}


Inserting (\ref{eq:3comp}) into (\ref{eq:press1}) and rearranging gives:

\begin{subequations}
\label{pv2}
  \begin{eqnarray}
    pV &=& \frac{1}{3} n N_A m \left < v^2  \right > = \frac{2}{3} n N_A 
             (\frac{1}{2} m \left < v^2 \right >) \label{eq:pv2a} \\
   &=& \frac{2}{3} n N_A \left < KE \right >
  \end{eqnarray}
\end{subequations}
where KE is the kinetic energy and we've just got the ideal gas law in disguise.

\par 

Substituting (\ref{eq:state}) into (\ref{eq:pv2a}) gives

\begin{subequations}
  \begin{eqnarray}
    \label{eq:pv3}
     \frac{2}{3} n N_A \left < KE \right > &=& n R^* T\label{eq:pv3a} \\
    \left < KE \right > &=& \frac{3}{2} \frac {R^*}{N_A} T = \frac{ 3}{2} kT
  \end{eqnarray}
\end{subequations}
$k$ is \textit{Boltzman's constant (again)}.
The smallest unit of thermal energy that our system can have
is $\frac{ 1}{2} k T$

According to \eqref{eq:pv3}, temperature is proportional to
the translational kinetic energy for this monotomic gas.  We've connected
thermal energy with Newton's 2nd law.  How does this
compare with other definitions?

\begin{itemize}

\item Temperature: (A measure of the tendency of an object to spontaneously
give up energy to its surroundings)
\item Heat: (Spontaneous flow of energy from one object to another
caused by a difference in temperature)
\item Work: (Any transfer of energy into or out of a system not due to temperature differences)

\end{itemize}

These three definitions are correct, but make no mention of the fact
that the gas is made of atoms.  This is the difference between
\textit{statistical mechanics} and \textit{classical thermodynamics}.

A final point:  we've only counted translational kinetic energy, because our atoms can't store
energy by rotating or vibrating.  We'll account for those two forms of internal energy
later on.


\end{document}


%%% Local Variables:
%%% mode: latex
%%% TeX-master: t
%%% End:
