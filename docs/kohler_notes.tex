\documentclass[12pt]{article}
\usepackage{xr,geometry,fancyhdr,hyperref,ifpdf,amsmath,rcs}
\usepackage{color,lastpage,longtable,Ventry,url,paunits,shortcuts,smallsec,float}
\geometry{letterpaper,top=50pt,hmargin={20mm,20mm},headheight=15pt} 

\pagestyle{fancy} 


\RCS $Revision: 1.1 $
\RCS $Date: 2002/01/11 03:49:21 $

\fancypagestyle{first}{
\lhead{\textbf{A405}}
\chead{\textbf{K\"ohler equation notes}} 
\rhead{p.~\thepage/\pageref{LastPage}}
\lfoot{} 
\cfoot{} 
\rfoot{}
}

\ifpdf
    \usepackage[pdftex]{graphicx} 
    \usepackage{hyperref}
    \pdfcompresslevel=0
    \DeclareGraphicsExtensions{.pdf,.jpg,.mps,.png}
\else
    \usepackage{hyperref}
    \usepackage[dvips]{graphicx}
    \DeclareGraphicsRule{.eps.gz}{eps}{.eps.bb}{`gzip -d #1}
    \DeclareGraphicsExtensions{.eps,.eps.gz}
\fi
\externaldocument[lec1,]{lec1}
\externaldocument[lec2,]{lec2}
\externaldocument[lec3,]{lec3}
\hyperbaseurl{http://www.eos.ubc.ca/courses/atsc405/}
\begin{document}
\pagestyle{first}

\begin{center}
Atsc405: K\"ohler equation\\
\end{center}


\textit{My own take on the K\"ohler equation  -- covering Thompkins section 4.2 and Lohmann Chapter 6 sections 6.1 - 6.4}

For a reversible process in a system with only liquid
water and water vapor, the Gibbs free energy/kg is given by:

\begin{equation}
  \label{eq:dg}
  dg= -\phi dT + v de
\end{equation}
where $v = 1/\rho$ is the volume per unit mass (specific volume)

It follows from (\ref{eq:dg}) and the equation of state (\textit{show
  this}) that for a pure sheet of flat water at constant temperature
we can integrate (\ref{eq:dg}) from an equilibrium state, where $e=e_s$,
and $g_v = g_l$,
to a point off the equilibrium curve, with $e=e_s + \delta e$,
and $g_v > g_l$
the energy difference between vapor and water is:

\begin{equation}
  \label{eq:gdiff}
  \delta g_l - \delta g_v = \delta (g_l - g_v) = 
( v_l - v_v ) \delta e \approx - \frac{R_v T}{e^\prime} \delta e^\prime
\end{equation}

We can integrate (\ref{eq:gdiff}) to get the tota energy difference
between vapor and liquid:

\begin{equation}
  \label{eq:totdiff}
  g_l - g_v \approx - \int_{e_s}^e \frac{R_v T}{e} de
= -R_v \ln \left ( \frac{ e}{e_s}  \right ) = -R_v T \ln (S)
\end{equation}
where S is the saturation ratio $e/e_s$.  Note that  if this is a
negative number (S $<$ 1) the system will lower its total G by moving
molecules from liquid to vapor, while if (S $>$ 1) the system will
move vapor to liquid.  If $S=1$ the system is in equilibrium and $g_l=
g_v$.  If there are $m_v$ kg of vapor and $m_l$ kg of liquid, then the
total Gibbs free energy G (J) is given by:

\begin{equation}
  \label{eq:total}
  G= m_v g_v + m_l g_l
\end{equation}
See if you can show that  this system will
maximize its entropy by minimizing G.

Here's a subtle question: How can we use equilibrium thermodynamics
when we're moving around off of the equilibrium curve?  I've acknowledged
this problem by writing the perturbation from equilibrium with
a $\delta$ instead of a $d$.  This $\delta$ is called a 
\textit{virtual perturbation} and it is meant to represent the fact
that we're imposing a new constraint (in the case of a cloud droplet,
we're putting it in an effectively infinite environment with a 
vapor pressure that is fixed to $e=e_s + \delta e$.  The environment
does whatever it takes to hold the vapor pressure away from the
drop at $e$, and this gives the system time to settle down
so that equilibrium statistics like $\left < v^2 \right >$, pressure,
temperature, etc. can be assumed to exist.


But what happens if the flat water is actually composed of $n_w$
molecules of pure water and $n_s$ ions of some solute (probably
$NH_4$ and $SO_4$)?  This lowers the equilibrium supersaturation through
Raoult's law (what I call the ``lilly pad effect''):

\begin{equation}
  \label{eq:raoult}
e_\chi = e_s \left ( \frac{n_w}{n_w + n_s} \right ) = a_w e_s  
\end{equation}

\noindent
Where the subscript $\chi$ (Chi) stands for ``chemistry''.  The more
solute ions ($n_s$) you have, the harder it is for the vapor to
escape, and the lower the equilibrium vapor pressure: the ``activity''
($a_w$) will be less than or equal to 1.

Here are some numbers to make (\ref{eq:raoult}) more concrete.  Consider
amonium sulphate $(NH_4)_2 SO_4$, with a molecular weight of
132 kg/Kmole.  This aerosol has a ``van't Hoff'' factor of $i=3$
(Lohmann, chapter 6, page 176) which means that it breaks up into 3 ions.  For
an aersol with a mass of $10^{-19}$ kg I get:

\begin{equation}
  \label{eq:aero}
  n_s = \frac{10^{-19}\ \un{kg} \cdot 3\ \mbox{ions/molecule}}%
     {0.132\ \un{kg/mole}} = 2.3 \times 10^{-18}\ \mbox{moles}
\end{equation}

Compare this with $n_w$ for a 1 \mum drop:

\begin{equation}
  \label{eq:water}
  n_w = \frac{\rho_l \cdot (4/3) \pi (10^{-6})^3}{0.018\ \mbox{kg/mole}}
  = 2.3 \times 10^{-13}\ \mbox{moles}
\end{equation}

So $n_s/n_w \approx 10^{-5}$.  Note however that as $r$ decreases
$n_s/n_w$ will increase as $1/r^3$.

Now suppose the liquid is composed of droplets of radius r,
with surface energy $4 \pi r^2  \sigma $ and therefore $G$ becomes:

\begin{equation}
  \label{eq:curved}
  G= m_v g_v + m_l g_l + 4\pi r^2 \sigma
\end{equation}
(Thompkings 4.5).  This new $G$ including the surface energy is called the
``thermodynamic potential''.
As before, it is
possible to show that the second law requires that our droplet/vapor
system minimize $G$.


Also suppose that $T$ and
$e$ are determined by the local environment (i.e. the temperature and
water vapor of the surrounding air) so that the system changes its
total thermodynamic potential only by changing the size of the droplets.
Then the differential of  (\ref{eq:curved}) can be written as:

\begin{equation}
  \label{eq:dgdr}
  \delta G = g_v \delta m_v + g_l \delta m_l + 8 \pi r \sigma \delta r
\end{equation}

\noindent
Note that $g_v$ and $g_l$ don't change because $e$ and $T$ are being
held constant.

Considering a solution instead of pure water gives a new lower limit
for the integration of~~(\ref{eq:totdiff}):  $a_w e_s$, so that:

\begin{equation}
  \label{eq:totdiffsol}
  g_l - g_v \approx -R_v T \ln \left (\frac{S}{a_w} \right )
\end{equation}



  We also
know that water is conserved so that 

\begin{equation}
  \label{eq:conserv}
dm_v = - dm_l =    - 4 \pi \rho_l r^2 dr
\end{equation}
Substituting (\ref{eq:conserv}) into (\ref{eq:dgdr}) and using (\ref{eq:totdiffsol})
yields:


\begin{equation}
  \label{eq:dEdr}
  \frac{\delta G}{\delta r} = \left [ -R_v T \rho_l 4 \pi r^2 \ln \left ( \frac{S}{a_w}
    \right ) + 8 \pi r \sigma \right ]
\end{equation}

\noindent
The K\"ohler curve consists of those values of droplet radius $r$ and
saturation $S$ such that $G$ is at an extremum.  Setting $\delta G/\delta r$=0 in (\ref{eq:dEdr})
and rearranging gives (you should show this):

\begin{equation}
  \label{eq:kohler}
  S= a_w \exp \left [ \frac{2\sigma}{\rho_l R_v T r} \right ]
\end{equation}
and from assignment 5 or Thompkins eq. 4.15:

\begin{equation}
  \label{eq:taylor}
S= a_w \exp \left [ \frac{2\sigma}{\rho_l R_v T r} \right ] \approx \left ( 1 + \frac{a}{r} - \frac{b}{r^3} \right )
\end{equation}
where
$a=\frac{2 \sigma}{\rho_l R_v T}$.
and  $b= \frac{i m M_w}{ 4/3\pi \rho_s M_s}$

The equilibrium supersaturation will be a minimum for droplet radius $r_{crit}$ that satisfies:

\begin{equation}
\label{eq:minS}
\frac{\partial S}{\partial r} = 0
\end{equation}
Take the derivative of (\ref{eq:taylor}) and show that this critical radius is given by

\begin{equation}
  \label{eq:rcrit}
  r_{crit}  = \left ( \frac{3b}{a} \right )^\frac{1}{2}
\end{equation}
Next week we'll work out $\frac{\delta ^2 G}{\delta r^2}$ 
and show that
it changes sign (switches from a stable to unstable
equilibrium when $r > r_{crit}$.
Figure \ref{fig:kohler} shows both the equilibrium K\"ohler curve (where $G$
is miniumum and contours of $G$

\begin{figure}
  \centering
\includegraphics[width=0.8\textwidth]{kohler_fig}
  \caption{K\"ohler equilibrium curves}
  \label{fig:kohler}
\end{figure}

The critical radius $r_{crit}$ 
distinguishes \textit{haze particles} from \textit{cloud droplets}.
Haze particles have $r < r_{crit}$ and are in stable equilibrium at
saturation $S$. Cloud droplets are \textit{activated} from haze
particles when they grow enough so that $r > r_{crit}$.  This is how
a thermal diffusion cloud chamber works (notes coming later this week)

This means that the equilibrium vapor curve given by (\ref{eq:kohler}) cannot
be used by itself to predict the vapor pressure for a population of
activated droplets with $r > r_{crit}$, because the unstable system will
continue to grow larger and larger droplets, lowering its total
E.  To track the growth of these droplets requires a more sophisticated
model which we will develop next.


\end{document}


%%% Local Variables:
%%% mode: latex
%%% TeX-master: t
%%% End:
