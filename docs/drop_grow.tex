\documentclass[12pt]{article}
\usepackage{geometry,fancyhdr,xr,hyperref,ifpdf,amsmath,rcs,shortcuts}
\usepackage{lastpage,longtable,color,paunits,amsmath}
\geometry{letterpaper,headsep=5mm,head=10mm,hmargin={20mm,30mm},bottom=10mm,top=5mm}


\pagestyle{fancy} 
\lhead{Atsc. 405} 
\chead{Notes on Droplet Growth} 
\rhead{page~\thepage/\pageref{LastPage}} 
\lfoot{} 
\cfoot{} 
\rfoot{}


\ifpdf
    \usepackage[pdftex]{graphicx} 
    \usepackage{hyperref}
    \pdfcompresslevel=0
    \DeclareGraphicsExtensions{.pdf,.jpg,.mps,.png}
\else
    \usepackage{hyperref}
    \usepackage[dvips]{graphicx}
    \DeclareGraphicsRule{.eps.gz}{eps}{.eps.bb}{`gzip -d #1}
    \DeclareGraphicsExtensions{.eps,.eps.gz}
\fi


\begin{document}

\begin{center}
{\bf
Atsc. 405\\
Notes on droplet growth\\
 }
\end{center}


\textit{The following note extend Lohmann chapter 7 to p. 198 and Thompkins Chapter 4}


We want to compute something called \textit{equilibrium} droplet
growth, i.e., what is the instantaneous droplet growth rate in an
environment in which the concentration of vapor molecules is changing
very slowly?  It is possible to show (see me if you care) that in a
binary mixture of gasses (water vapor and nitrogen) the water vapor
mass flux $F_m\ (\un{kg\,m^{-2}\,s^{-1}})$ is given by:

\begin{equation}
  \label{eq:ficks1}
  F_m = -D\, m_w\, n\, \nabla \left (\frac{n_v}{n} \right )
\end{equation}
where $m_w$ is the weight of an individual water molecule, $n_v$ is
the number density of the vapor (\un{vapor\ molecules\,m^{-3}}),
$n$ is the total number density for all gas molecules
and  $D \approx
2.2 \times 10^{-5}~\un{m^2\,s^{-1}}$ is the diffusivity of water vapor.
In words (\ref{eq:ficks1}) says that it is spatial variations in
the number mixing ratio $n_v/n$ (and not, for example, the vapor
density $\rho_v$) that drive diffusion.

We also know from the equation of state that the total number density
$n$ and the vapor molecule density $n_v$ are related to the temperature by:

\begin{eqnarray}
  \label{eq:state0}
  p \alpha &=& n k_b T\nonumber\\
  e &=& m_w\, n_v\, R_v\, T
\end{eqnarray}
where $p$  is the total pressure, $\alpha$ is the specific volume (\un{m^3\,kg^{-1}}),
$n$ is defined as above and $k_b$ is Boltzman's constant (the gas constant per molecule).

Consider an environment in which the total pressure $p$, volume $V$
and mass (and therefore total density $\rho$ and specific volume
$1/\rho = \alpha$) are all independent of position (for example, in a
laboratory chamber at uniform pressure).  Then insert (\ref{eq:state0}) into
(\ref{eq:ficks1}) and get (do this):

\begin{equation}
  \label{eq:goodeqn}
  F_m = -\frac{D}{R_v\,T} \nabla e
\end{equation}

Can we write (\ref{eq:goodeqn}) in terms of the vapor density 
$\rho_v = m_w\,n_v$?
Remember that $p\,\alpha \propto T$, so that from (\ref{eq:state0})
we have $n \propto 1/T$.  Using the chain rule on (\ref{eq:ficks1})
I get:

\begin{equation}
  \label{eq:ficks2}
  F_m = -D\,n\, \nabla \left (\frac{m_w\,n_v}{n} \right ) =
 -D\,n\,\left ( \frac{\nabla \rho_v}{n} - \frac{\rho_v}{n^2} \nabla n \right )
= -D \left ( \nabla \rho_v + \frac{\rho_v \nabla T}{T} \right )
\end{equation}

It turns out that many people (including Thompkins and Lohmann) write down the diffusion
equation as:

\begin{equation}
  \label{eq:ficks3}
  F_m  = -D  \nabla \rho_v 
\end{equation}
Comparing (\ref{eq:ficks3}) with (\ref{eq:ficks2}) shows that
this is equivalent to assuming $\nabla T=0$, which isn't
quite true for a growing cloud drop.  If we can accept and error of 10\% or 
we can neglect this complication and 
and use (\ref{eq:ficks3}).


We also have a similar expression for the diffusion of energy
(i.e.~enthalpy = $c_p T$):

\begin{equation}
  \label{ficksheat}
  F_Q~(\un{J\,m^{-2}\,s^{-1}})= -K \nabla T
\end{equation}

\noindent 
where $K \approx 2.2 \times 10^{-2}~\un{J\,m^{-1}\,s^{-1}\,K^{-1}}$ is
the thermal conductivity of dry air and we're neglecting the
difference between the heat capcities and diffusivities of water and
nitrogen.  Both $D$ and $K$ are also weak functions of temperature and
pressure, which we'll also ignore.  The minus signs in
(\ref{eq:ficks3}) and (\ref{ficksheat}) indicate that when $n_v$, for
example, increases to the right, the flux is \emph{down-gradient} to
the left.

As shown in class, in one dimension we can get the time rate of change of the
density by taking the derivative of the flux (why?):

\begin{equation}
  \label{old}
  \frac{\partial \rho_v}{\partial t} = - \frac{ dF_m}{dx}
\end{equation}

\noindent
In three dimensions this becomes:

\begin{equation}
  \label{divergence}
  \frac{\partial \rho_v}{\partial t} = - \nabla F_m = D m_w \nabla^2 n_v
\end{equation}
and

\begin{equation}
  \label{divergenceT}
  \frac{\partial T}{\partial t} = - \nabla F_Q = K \nabla^2 T
\end{equation}


\noindent
where we have small the variation in $D$ with $T$.

Now take a look at what $\nabla^2 n_v$ looks like in spherical
coordinates:

\begin{equation}
  \label{spherical}
  \nabla^2 n_v = \frac{1}{R^2} \frac{\partial}{\partial R} \left ( 
               R^2 \frac{\partial n_v}{\partial R} \right )
             + \frac{1}{R^2 \mathrm{sin \theta}}
             \frac{\partial}{\partial \theta} 
              \left (\mathrm{sin \theta} \frac{\partial n_v}{\partial \theta} \right )
             + \frac{1}{R^2 \mathrm{sin \theta}}
             \frac{\partial^2 n_v}{\partial \phi^2} 
\end{equation}

Equation~(\ref{spherical}) simplifies considerably, because we know
that $\frac{\partial^2 n_v}{\partial \phi^2}=\frac{\partial
  n_v}{\partial \theta}=0$ for droplets (why?).  Also, if we assume
that the droplet is growing in equilibrium, then even though it is
continuously using up water vapor, there is always more water vapor
available energy availble, and so 
$\frac{\partial n_v}{\partial t} = \frac{\partial T}{\partial t} =0$. This is 
what is meant by equilbrium in the phrase \textit{equilibrium droplet growth}.
(The other part is that $T$ of the environment around the droplet is
also assumed constant).

Using (\ref{spherical}) we have new equations for the vapor and
temperature fields via (\ref{divergence}) and (\ref{divergenceT}):

\begin{eqnarray}
  \label{duel}
  \frac{\partial n_v}{\partial t}& = & D \nabla^2 n_v = D \frac{1}{R^2}
               \frac{\partial}{\partial R} \left ( R^2 \frac{\partial
               n_v}{\partial R} \right ) = 0 \nonumber \\
  \frac{\partial Q}{\partial t}& = &  K \nabla^2 T = c_p K \frac{1}{R^2}
               \frac{\partial}{\partial R} \left ( R^2 \frac{\partial
               T}{\partial R} \right ) = 0
\end{eqnarray}

To solve (\ref{duel}) you need to integrate it twice wrt R (do this).
You should find that you get solutions of the form:

\begin{eqnarray}
  \label{soln}
  n_v(R) &=& c_1 - \frac{c_2}{R} \nonumber\\
  T(R) &=& c_3 - \frac{c_4}{R}
\end{eqnarray}

To determine the coefficients, recognize that we know the values of
$n_v$ and $T$ at $R=\infty$ (say 100 droplet radii away from the
surface) where they are just determined by the bulk $\rho_v$ and $T$
of the cloudy air, without any influence due to surface curvature or
solution effects.  At the droplet surface $R=r$, $n_v$ will be
determined from the K\"ohler curve: 

\begin{equation}
  \label{rhovr}
   n_v(r)= \rho_v(r)/m_w= \frac{e_r}{R_v T(r)}=\frac{e_s(T(r))(1 + a/r -
     b/r^3)}{R_v T(r)}
\end{equation}

Writing $n_r=n_v(r)$ and $T_r=T(r)$:

\begin{eqnarray}
  \label{soln2}
  n_v(R) &=& n_{v\,\infty} - \frac{r}{R}(n_{v\,\infty} - n_r) \nonumber\\
  T(R) &=& T_\infty - \frac{r}{R}(T_\infty - T_r)
\end{eqnarray}

In equilibrium it will turn out that $n_v$ will increase, and $T$ will
decrease as we move out from the surface of the drop, so that vapor
can move toward the drop, and the heat of condensation can be
conducted away from the drop.

In spherical coordinates with $\theta$ and $\phi$ symmetry we have
only radial dependence, so that:

\begin{equation}
  \label{symm}
  \nabla n_v = \frac{\partial n_v}{\partial R}
\end{equation}

Putting (\ref{soln2}) into (\ref{eq:ficks3}) gives:

\begin{equation}
  \label{fluxes}
  F_m = -D m_w \frac{\partial n_v}{\partial R} = -D m_w ( \frac{r}{R^2}
  (n_\infty - n_r))
\end{equation}

\noindent
and for the temperature:

\begin{equation}
  \label{fluxes2}
  F_T = -K \frac{\partial T}{\partial R} = -K ( \frac{r}{R^2}
  (T_\infty - T_r))
\end{equation}

But what happens to all of this water?  It goes into making the
droplet bigger, because at some point it hits the surface of the
droplet and condenses.  Thus we have (using (\ref{fluxes}) at $R=r$):

\begin{eqnarray}
  \label{final}
  \frac{dm}{dt}& =& -4 \pi r^2 m_w \left ( -D \left [ \frac{n_{v\,\infty} - n_r}{r} \right ] \right ) 
        \nonumber\\ 
  \frac{dm}{dt}& = &4 \pi r D (\rho_{v \infty} - \rho_{v r})
\end{eqnarray}

\noindent
which is just Thompkins 4.19 (where like them we've assumed no gradient
in $T$).

You should repeat this for the temperature and show that the flux of
enthalpy away from the droplet is just:


\begin{equation}
  \label{heatflux}
  \frac{dH}{dt}~(\un{J\,s^{-1}})= 4 \pi r K (T_r - T_\infty)
\end{equation}

Now all of the vapor that is condensing onto the growing droplet has
to be diffused away, so balancing (\ref{heatflux}) with (\ref{final})
gives a relationship between the vapor density and the temperature at
the surface of the droplet:

\begin{eqnarray}
  \label{balance}
  L \frac{dm}{dt} - \frac{dH}{dt} &=& 0 ~\mathrm{~~~~~~~or:} \nonumber\\
  4 \pi  r D L ( \rho_{v \infty} - \rho_{v r})&=&  4 \pi r K (T_r
  - T_\infty )
\end{eqnarray}


Taken together, we now have a closed system, with five unknowns
($m$, $\rho_{v r}$, $\rho_{v \infty}$, $T_r$, $T_\infty$) and five
equations (\ref{rhovr}), (\ref{final}), (\ref{balance}), and equations for
the far-field $T_\infty$ and $\rho_{v \infty}$: 

\begin{eqnarray}
  \label{moistad}
  \frac{dT_\infty}{dt}& =& \left [ \frac{ -g}{c_p + L \frac{d w_{v \infty}}{dt}}
     \right ] \frac{dz}{dt}~~~~~~~~~~~(\mathrm{moist~adiabat})\nonumber\\
  \frac{dw_{v \infty}}{dt}& =& -\frac{dw_{l}}{dt}= -\frac{4 \pi}{3}
   \frac{\rho_l}{\rho_a} \int_0^\infty \eta(r) r^2 \frac{dr}{dt} dr~~~~~~~~~~~(\mathrm{conservation~of~water})
\end{eqnarray}

\noindent
where $\rho_l$ and $\rho_a$ are the densities of water and dry air and
$\eta (r)~(\# m^{-3} \mu m^{-1})$ represents the number distribution
of droplets, that is $\eta (r) dr$ gives the droplet concentration of
droplets with radii between $r$ and $r + dr$.  Remember that we need
to convert the droplet concentration from ($kg\,m^{-3}$) to
($kg\,kg^{-1}$).  We can integrate $\eta(r)$ over all $r$ to get the
total number:

\begin{equation}
  \label{integ}
  N_{total} = \int_0^\infty \eta(r) dr
\end{equation}


Remember also that we can get $\rho_{v \infty}$ from $w_{v \infty}$ through
the definition:

\begin{equation}
  \label{definition}
  w_{v \infty}= \frac{\rho_{v \infty}}{\rho_{d \infty}} = \frac{e/(R_v T_\infty)}{\rho_{d \infty}}
\end{equation}
where $\rho_{d \infty}$, the density of dry air for the environment, can
be found from the equation of state, as long as we specify the pressure
and find the vapor pressure $e$, from 

\begin{equation}
  \label{eq:state}
  \rho_{d \infty} = \frac{p - e}{R_d T_\infty}
\end{equation}

\includegraphics{rplot.png}


\includegraphics{splot.png}

\end{document}

%%% Local Variables:
%%% mode: latex
%%% TeX-master: t
%%% End:
